\documentclass{article}
\usepackage{times}

\usepackage{amsmath,amsthm,amssymb}
\usepackage{stmaryrd}
%\usepackage{mdsymbol}
\usepackage{MnSymbol}
\usepackage{xcolor}
\usepackage{verbatim}
\usepackage{array}


\title{Events, probability measure, conditional probability.}
\author{Antoine Venant}
\begin{document}
\maketitle


\section*{Exercice 1}

Propose a sample space $\Omega$ sufficient to formally express the listed events, and use it to formally express the listed events, in each of the following cases:

\begin{enumerate}
\item One throw two undistinguishable dice.
  \begin{enumerate}
  \item $e_1: $ The sum of the points of the two dice is even.
  \item $e_2: $ The second die has more points than the first.
  \end{enumerate}

\item $n$ different socks are distruted over $r$ different drawers (so some drawers can possibly be empty, a drawer could contain all of the $n$ socks)
  \begin{enumerate}
  \item $e_1: $ There is at least one drawer that contains at least two socks.
  \item $e_2: $ Every drawer contains at most one sock.
  \end{enumerate}
\item An old car has a chance to break every time someone uses it.
  \begin{enumerate}
  \item $e_1: $ The car breaks the third time we use it.
  \item $e_2: $ The car does not eventually break, \emph{i.e.} one successfully uses it infinitely often. 
  \end{enumerate}
\end{enumerate}



\end{document}
