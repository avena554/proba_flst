\documentclass{article}
\usepackage{times}

\usepackage{amsmath,amsthm,amssymb}
\usepackage{stmaryrd}
%\usepackage{mdsymbol}
\usepackage{MnSymbol}
\usepackage{xcolor}
\usepackage{verbatim}
\usepackage{array}


\title{Extra exercises.}
\author{Antoine Venant}
\begin{document}
\maketitle

\section*{Exercise 1}
Let $e_1 = (1,0) \in \mathbb{R}^2$, $e_2 = (0,1) \in \mathbb{R}^2$. Let $\mathcal{E} = \langle (1,0), (0,1) \rangle$ be the canonical basis of $\mathbb{R}^{2}$.
Let also $v_1 = (2,2) \in \mathbb{R}^2$ and $v_2 = (4,1) \in \mathbb{R}^2$. Let $\mathcal{B} = \langle v_1, v_2 \rangle = \langle (2,2), (4,1) \rangle$. 
\begin{enumerate}
\item Show that $\mathcal{B}$ is a basis of $\mathbb{R}^2$.
\item Give the representation of the identity w.r.t. $(\mathcal{B}, \mathcal{E})$, \emph{i.e.}, the matrix $id_{\mathbb{R}^2} \restriction_{(\mathcal{B}, \mathcal{E})}$ whose columns respectively are $v_1 \restriction_{\mathcal{E}}$ and $v_2 \restriction_{\mathcal{E}}$.
\item Use $id_{\mathbb{R}^2} \restriction_{(\mathcal{B}, \mathcal{E})}$ to find the vector $u$ such that $u = \langle 1,1 \rangle_{\mathcal{B}}$.
\item  Find the 4 real numbers $a, a', b, b'$ such that 
  \[ e_1 = \langle a, a' \rangle_{\mathcal{B}}\textnormal{, } e_2 = \langle b, b' \rangle_{\mathcal{B}} \]
\item Give the representation of the identity w.r.t. $(\mathcal{E}, \mathcal{B})$, \emph{i.e.} the matrix $id_{\mathbb{R}^2} \restriction_{(\mathcal{E}, \mathcal{B})}$.
\item Use $id_{\mathbb{R}^2} \restriction_{(\mathcal{E}, \mathcal{B})}$ to compute $(6,3) \restriction_{\mathcal{B}}$.
\item What is the inverse matrix of $id_{\mathbb{R}^2} \restriction_{(\mathcal{B}, \mathcal{E})}$? Justify your answer (several justifications are possible, give the one you prefer).
\end{enumerate}

\section*{Exercise 2}
Let $A = \begin{pmatrix} a_{1,1} & \dots & a_{1,m} \\ \vdots &  & \vdots \\ a_{n,1} & \dots & a_{n,m} \end{pmatrix}$ be an $(n,m)$ matrix. We let $A[i] = (a_{i,1}, \dots, a_{i,m}) \in \mathbb{R}^m$ denote the $i^{th}$ line of $A$. 

\begin{enumerate}
\item Give a matrix $P_{\lambda}$ such that $P_{\lambda} \times A$ is the matrix obtained by repcaling $A[i]$ with $\lambda A[i]$ in $A$.
\item Give a matrix $Q_{\lambda, i, j}$ such that $P \times A$ is the matrix obtained by replacing $A[j]$ with $A[j] - \lambda A[i]$ in $A$.
\item Give a matrix $Q_{\neg i}$ such that $P \times A$ is the matrix obtained by replacing $A[j]$ with $A[j] - \lambda A[i]$ in $A$ \textbf{for every $j \neq i, j \in [1,n]$}.
\end{enumerate}

\section*{Exercice 3}
Let $A= \begin{pmatrix} a & b \\ c & d \end{pmatrix}$ be a $(2,2)$ matrix.
\begin{enumerate}
\item At which condition on $a,b,c,d$ are the column vectors $(a,b)$ and $(c,d)$ linearly independent?
\item At which condition on $a,b,c,d$ is $A$ invertible? Give two example $(2,2)$ matrices that are not invertible. 
\item Assume that $ad-bc \neq 0$ and (for simplicity) that $a \neq 0$. Use Gauss elimination to reduce $A$ to identity. Using Exercise $2$, identify each operation performed in the Gauss elimination algorithm with multiplication by a matrix on the left. Gauss elimination should require $4$ step, so you should find $4$ matrices $P_1, P_2, P_3, P_4$ such that $P_4 \times P_3 \times P_2 \times P_1 \times A = \begin{pmatrix}1 & 0 \\ 0 & 1 \end{pmatrix}$.
\item Under same assumption as above, give the inverse of $A$.
\end{enumerate}



\section*{Exercice 3}

We let $\mathbb{R}^{\mathbb{N}}$ denote the set of infinite sequences of real numbers. We can equip $\mathbb{R}^{\mathbb{N}}$ with pointwise addition ${+}$ and scalar multiplication ${\cdot}$ as follows: for a sequence $\langle U_n \rangle_{n \in \mathbb{N}} = \langle U_1, U_2, \dots \rangle$ we let:

\[ \lambda \cdot \langle U_n \rangle_{n \in \mathbb{N}} = \langle \lambda U_n \rangle_{n \in \mathbb{N}} \] \emph{i.e.} $\lambda \cdot \langle U_n \rangle_{n \in \mathbb{N}}$ is the sequence $\langle \lambda U_1, \lambda U_2, \dots \rangle$.

and for two sequences $\langle U_n \rangle_{n \in \mathbb{N}} = \langle U_1, U_2, \dots \rangle$ and $\langle V_n \rangle_{n \in \mathbb{N}} = \langle V_1, V_2, \dots \rangle$ we let \[\langle U_n \rangle_{n \in \mathbb{N}} + \langle V_n \rangle_{n \in \mathbb{N}} = \langle U_n + V_n \rangle_{n \in \mathbb{N}}.\] \emph{i.e.} $\langle U_n \rangle_{n \in \mathbb{N}} + \langle V_n \rangle_{n \in \mathbb{N}}$ is the sequence $\langle U_1 + V_1, U_2 + V_2, \dots \rangle$.

\paragraph{Example:} if for instance \[ \langle U_n \rangle_{n \in \mathbb{N}} = \langle 2n \rangle_{n \in \mathbb{N}} =  \langle 0, 2, 4, 6, \dots \rangle \]
and \[ \langle V_n \rangle_{n \in \mathbb{N}} = \langle 1 \rangle_{n \in \mathbb{N}} =  \langle 1, 1, 1, 1, \dots \rangle \] then

\[\begin{aligned}
&\frac{1}{2} \cdot \langle U_n \rangle_{n \in \mathbb{N}} = \langle n \rangle_{n \in \mathbb{N}} = \langle 0, 1, 2, 3, \dots \rangle\\
&\langle U_n \rangle_{n \in \mathbb{N}} + \langle V_n \rangle_{n \in \mathbb{N}} = \langle 2n + 1 \rangle_{n \in \mathbb{N}} = \langle 1, 3, 5, 7, \dots \rangle
\end{aligned}\]

\begin{enumerate}
\item Show that $\langle \mathbb{R}^{\mathbb{N}}, {+}, {\cdot}  \rangle$ is a vector space.
\end{enumerate}

An infinite sequence of real numbers $\langle U_n \rangle_{n \in \mathbb{N}}$ is a \emph{Fibonacci sequence} iff it verifies the following property:
\[ \forall n \in \mathbb{N}\, U_{n} + U_{n+1} = U_{n+2} \]
We let $\textsf{Fib} \subseteq \mathbb{R}^{\mathbb{N}}$ denote the set of Fibonacci sequences.

\begin{enumerate}
\item Show that $\textsf{Fib}$ is a vector subspace of $\mathbb{R}^{\mathbb{N}}$.
\item Find a $(2,2)$ matrix $M$ such that, for any Fibonacci sequence $\langle U_n \rangle_{n \in \mathbb{N}} \in \textsf{Fib}$ we have \[\forall n \in \mathbb{N}\, \begin{pmatrix} U_{n+2} \\ U_{n+1} \end{pmatrix} = M \times \begin{pmatrix} U_{n+1} \\ U_{n} \end{pmatrix}  \]
\item Deduce from the above that for any Fibonacci sequence $\langle U_n \rangle_{n \in \mathbb{N}}$, we have  \[ \forall n \in  \mathbb{N}\,\begin{pmatrix} U_{n+1} \\ U_n \end{pmatrix} = M^n \times \begin{pmatrix} U_{1} \\ U_{0} \end{pmatrix}\] where $M^n = \underbrace{M \times \dots \times M}_{n \textnormal{ times}}$ and we use the convention $M^0 = Id_{2}$ ($M^0$ is the identity $(2,2)$ matrix).
\item We let $\begin{pmatrix} A_n & B_n\\C_n & D_n \end{pmatrix} = M^n$. We consider the two following sequences of real numbers:
  \[ C = \langle C_n \rangle_{n \in \mathbb{N}} \textnormal{ and } D = \langle D_n \rangle_{n \in \mathbb{N}}.\] Let $U = \langle U_n \rangle_{n \in \mathbb{N}}$ be a Fibonacci sequence. Find $\lambda, \mu \in \mathbb{R}$ such that
  \[ U = \lambda \cdot C + \mu \cdot D. \] Conclude that $dim(\textsf{Fib}) \le 2$.
\item We let $v_1 = (\frac{1 + \sqrt{5}}{2}, 1) \in \mathbb{R}^2$ and $v_2 = (\frac{1-\sqrt{5}}{2}, 1) \in \mathbb{R}^2$. We consider the familly of vectors $\mathcal{B} = \langle v_1, v_2 \rangle$. Show that $\mathcal{B}$ is a basis of $\mathbb{R}^2$.
\item Verify that $(1, 0) = \langle \frac{1}{\sqrt{5}}, -\frac{1}{\sqrt{5}} \rangle_{\mathcal{B}}$ and $(0, 1) = \langle \frac{\sqrt{5}-1}{2\sqrt{5}}, \frac{\sqrt{5} + 1}{2\sqrt{5}}  \rangle_{\mathcal{B}}$.
\item Let $\mathcal{E} = \langle (1,0), (0,1) \rangle$ be the canonical basis of $\mathbb{R}^{2}$. Give the matrices $P = id_{\mathbb{R}^2} \restriction_{(\mathcal{B}, \mathcal{E})}$ and $Q = P^{-1} = id_{\mathbb{R}^2} \restriction_{(\mathcal{E}, \mathcal{B})}$. Justify briefly why $Q = P^{-1}$.
\item Compute the matrix $D = P^{-1} M P$. You should obtain a diagonal matrix (\emph{i.e.}, something with $0$s everywhere except on the diagonal). Show that we have \[ M = P \times D \times P^{-1}\]
\item Establish that $M^n = P D^n P^{-1}$.
\item Express the $4$ coefficient of $D^n$ as a function of $n$, then use the result to show that \[ M^n = \begin{pmatrix} \frac{1}{\sqrt{5}}\left ( (\frac{1+\sqrt{5}}{2})^{n+1} - (\frac{1-\sqrt{5}}{2})^{n+1} \right ) & \frac{1}{\sqrt{5}} \left ( \frac{\sqrt{5} - 1}{2} (\frac{1+\sqrt{5}}{2})^{n+1} + \frac{1+\sqrt{5}}{2} (\frac{1-\sqrt{5}}{2})^{n+1}    \right ) \\ \frac{1}{\sqrt{5}}\left ( (\frac{1+\sqrt{5}}{2})^{n} - (\frac{1-\sqrt{5}}{2})^{n} \right ) & \frac{1}{\sqrt{5}} \left ( \frac{\sqrt{5} - 1}{2} (\frac{1+\sqrt{5}}{2})^{n} + \frac{1+\sqrt{5}}{2} (\frac{1-\sqrt{5}}{2})^{n}    \right )    \end{pmatrix} \].
\item Using question 4. conclude that $U = \langle U_n \rangle_{n \in \mathbb{N}}$ is a Fibonacci sequence iff for all $n \in \mathbb{N}$,
  \[ U_n =  U_1 \left [ \frac{1}{\sqrt{5}}\left ( \left (\frac{1+\sqrt{5}}{2} \right )^{n} - \left (\frac{1-\sqrt{5}}{2} \right )^{n} \right ) \right ] + U_0 \left [ \frac{1}{\sqrt{5}} \left ( \frac{\sqrt{5} - 1}{2} \left (\frac{1+\sqrt{5}}{2} \right )^{n} + \frac{1+\sqrt{5}}{2} \left (\frac{1-\sqrt{5}}{2} \right )^{n}    \right )  \right ].  \]
\end{enumerate}







\end{document}
