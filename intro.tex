\documentclass{beamer}

%\usepackage{listings}
%\usepackage[francais]{babel}
\usepackage[T1]{fontenc}
\usepackage[utf8]{inputenc}
%\usepackage{MyriadPro}
\usepackage{cabin}
\usepackage{graphicx}
\usepackage{array}
\usepackage{tikz}
\usetikzlibrary{positioning, backgrounds, shapes, chains, decorations.pathmorphing}
\usepackage{amsmath,amsthm,amssymb}  
\usepackage{stmaryrd}
%\usepackage{mdsymbol}
\usepackage{MnSymbol}
\usepackage{xcolor}
\usepackage{verbatim}
\usepackage{array}
%\usepackage{csquotes}


\useoutertheme{infolines}

\newcommand{\hidden}[1]{}

%colors
\definecolor{darkgreen}{rgb}{0,0.5,0}
\usebeamercolor{block title}
\definecolor{beamerblue}{named}{fg}
\usebeamercolor{alert block title}
\definecolor{beamealert}{named}{fg}

\renewcommand{\colon}{\!:\!}


\newcommand\paraitem{%
 \quad
 \makebox[\labelwidth][r]{%
 \makelabel{%
 \usebeamertemplate{itemize \beameritemnestingprefix item}}}\hskip\labelsep}

\newcommand{\mmid}{\mathbin{{\mid}{\mid}}}

\begin{document}

\title{Probability Theory -- Introduction} 
\author{Antoine Venant}
%\institute{UDS COLI}
\date{\today}
\maketitle


\begin{frame}
  \frametitle{Probabilities}
  
  \begin{block}{Nondeterminism}
    \begin{itemize}
    \item \emph{Random} experiments or trials: `repeated' trials yield different results.
    \item Typical examples: throwing a dice, flipping a coin. 
    \item You might or might not believe in randomness: even as a view of the mind it allows approximating things at a manageable level.
    \item Allows drawing `global' conclusions without having to know everything.
    \end{itemize}
  \end{block}

  \begin{exampleblock}{Example}
    I throw a fair $6$-faced dice a million times.
    \begin{itemize}
    \item What can I expect about the value of the $10000$th throw?
    \item What can I expect about the average score?
    \end{itemize}
  \end{exampleblock}

\end{frame}

\begin{frame}
  \begin{exampleblock}{Example.}
    Would you accept playing the following game with me:
  \end{exampleblock}
  
  \begin{block}{Modelling uncertainty and incomplete knowledge.}
    
  \end{block}
  
\end{frame}

\begin{frame}
  \frametitle{Probabilities and Language.}
  - Bending the rules.
  - Ambiguity
  - Zipf stuff?
\end{frame}

\begin{frame}{Mathematical Foundations: Equiprobable events.}
  Let's start with some examples.
\end{frame}

\begin{frame}{Sample space, counting.}
\end{frame}

\begin{frame}{Non equiprobable events?}

\end{frame}

\end{document}
