\documentclass{beamer}

%\usepackage{listings}
%\usepackage[francais]{babel}
\usepackage[T1]{fontenc}
\usepackage[utf8]{inputenc}
%\usepackage{MyriadPro}
\usepackage{cabin}
\usepackage{graphicx}
\usepackage{array}
\usepackage{tikz}
\usetikzlibrary{positioning, backgrounds, shapes, chains, decorations.pathmorphing}
\usepackage{amsmath,amsthm,amssymb}  
\usepackage{stmaryrd}
%\usepackage{mdsymbol}
\usepackage{MnSymbol}
\usepackage{xcolor}
\usepackage{verbatim}
\usepackage{array}
%\usepackage{csquotes}


\useoutertheme{infolines}

\newcommand{\hidden}[1]{}

%colors
\definecolor{darkgreen}{rgb}{0,0.5,0}
\usebeamercolor{block title}
\definecolor{beamerblue}{named}{fg}
\usebeamercolor{alert block title}
\definecolor{beamealert}{named}{fg}

\renewcommand{\colon}{\!:\!}


\newcommand\paraitem{%
 \quad
 \makebox[\labelwidth][r]{%
 \makelabel{%
 \usebeamertemplate{itemize \beameritemnestingprefix item}}}\hskip\labelsep}

\newcommand{\mmid}{\mathbin{{\mid}{\mid}}}

\begin{document}

\title{Probability Theory -- Introduction} 
\author{Antoine Venant}
%\institute{UDS COLI}
\date{\today}
\maketitle

\begin{frame}
  \frametitle{What is probability theory?}

  Probability theory is the branch of mathematics that deals with throwing dices and flipping coins (and a few other things too).\\

  \begin{exampleblock}{Example}
    I throw a fair $6$-faced dice a million times. Can I say anything interesting \emph{a priori}:
    \begin{itemize}
    \item About the outcome of the $10000$th throw?
    \item About the number of times i'll get a $6$?
    \item About the average of all numbers I got?
    \end{itemize}
  \end{exampleblock}

  
\end{frame}


\begin{frame}
  \frametitle{Why a theory of probabilities?}
  
  \begin{block}{Nondeterminism}
    \begin{itemize}
    \item Modelling \emph{Random} phenomena or \emph{trials}: when `repeated' trials yield different outcomes, \emph{i.e.} the outcome \emph{fluctuates}.
    \item Typical examples from before: throwing a dice, flipping a coin.
    \item But not only: playing roulette, running a processor at a certain speed (may crash or not), having $n$ radioactive element disintegrate in a given amount of time, transmitting a message through a faulty device...
    \item Draw conclusions, make prediction about outcomes and how much they can \emph{fluctuate}.
    \end{itemize}
  \end{block}  
\end{frame}

\begin{frame}
  \frametitle{Why a theory of probabilities (contd)?}
  Which of the following choice would you say is more interesting:
  
  \begin{exampleblock}{Choice 1.}
    I give you 1500 euro.
  \end{exampleblock}

  \begin{exampleblock}{Choice 2.}
    We play 1000 times the following game:
    \begin{enumerate}
    \item you throw one $6$-faced dice.
    \item If the result is $1$ or $2$, $3$ or $4$ you give me $1$ euro and the game stops.
    \item If the result is $5$ or $6$, I give you $4$ euros, but we play again.
    \end{enumerate}
  \end{exampleblock}

\end{frame}

\begin{frame}
  
  \begin{block}{Uncertainty and incomplete knowledge.}
    \begin{itemize}
    \item Probabilities as a mean to measure \emph{degrees} of uncertainty.
    \item Guide rational decisions under uncertainty (work of Von Neunmann, Nash, \dots).
    \item No need to even \emph{believe} in non-determinism:
    \item probabilities as a view of the mind abstracting over unknown/too numerous/unmeasurable factors influencing a deterministic outcome.
    \item getting a simple and manageable approximation, allowing to draw conclusions and make predictions.
    \end{itemize}
  \end{block}
 
\end{frame}

\begin{frame}
  \frametitle{Probabilities and language.}
  For a thorough treatment of this topic see [Manning \& Sch\"utze, \emph{Foundations of Natural Language Processing.}]. Here are a couple points borrowed from there: 

  \begin{block}{Different possible expressions are not equally likely}
    Language use is best described accounting for the \emph{likelihood} of people to chose an expression over another.
  \end{block}

  \begin{exampleblock}{Convention}
    \begin{itemize}
    \item[a)] I did not shoot the deputy.
    \item[b)] I did not fire my gun on the deputy and peirce him with a bullet.
    \end{itemize}
  \end{exampleblock}
\end{frame}

\begin{frame}
  \frametitle{Probabilities and language (contd).}
  
  \begin{block}{Degrees vs. categorical judgements.}
    In some cases, binary linguistics judgmnents (veracity, grammaticality) are hard to carry, and easier to model on a continuous scale.
    \begin{itemize}
    \item what are truth conditions for adjectives like \emph{bald} or \emph{red}?
    \end{itemize}
    {\bf Moreover, \emph{continuous} models such as offered by probability theory have exploitable mathematical property that discrete models lack.}
  \end{block}
\end{frame}

\begin{frame}
  \frametitle{}
  \begin{block}{Ambiguity.}
    Ambiguity is present at every level of language, so that systematically asking for disambiguisation, or always entertaining all options would not be manageable.
  \end{block}
\end{frame}

\begin{frame}{Mathematical Foundations: Equiprobable events.}
  Let's start with some examples.
\end{frame}

\begin{frame}{Sample space, counting.}
\end{frame}

\begin{frame}{Non equiprobable events?}

\end{frame}

\end{document}
