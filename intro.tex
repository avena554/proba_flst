\documentclass{beamer}

%\usepackage{listings}
%\usepackage[francais]{babel}
\usepackage[T1]{fontenc}
\usepackage[utf8]{inputenc}
%\usepackage{MyriadPro}
\usepackage{cabin}
\usepackage{graphicx}
\usepackage{array}
\usepackage{tikz}
\usetikzlibrary{positioning, backgrounds, shapes, chains, decorations.pathmorphing}
\usepackage{amsmath,amsthm,amssymb}  
\usepackage{stmaryrd}
%\usepackage{mdsymbol}
\usepackage{MnSymbol}
\usepackage{xcolor}
\usepackage{verbatim}
\usepackage{array}
%\usepackage{csquotes}



\usepackage[absolute,overlay]{textpos}
%\usepackage[texcoord,
%grid,gridcolor=red!10,subgridcolor=green!10,gridunit=pt]
%{eso-pic}



\useoutertheme{infolines}

\newcommand{\hidden}[1]{}

%colors
\definecolor{darkgreen}{rgb}{0,0.5,0}
\usebeamercolor{block title}
\definecolor{beamerblue}{named}{fg}
\usebeamercolor{alert block title}
\definecolor{beamealert}{named}{fg}

\renewcommand{\colon}{\!:\!}


\newcommand\paraitem{%
 \quad
 \makebox[\labelwidth][r]{%
 \makelabel{%
 \usebeamertemplate{itemize \beameritemnestingprefix item}}}\hskip\labelsep}

\newcommand{\mmid}{\mathbin{{\mid}{\mid}}}

\begin{document}

\title{Probability Theory -- Introduction} 
\author{Antoine Venant}
%\institute{UDS COLI}
\date{\today}
\maketitle

\begin{frame}
  \frametitle{What is probability theory?}

  Probability theory is the branch of mathematics that deals with throwing dice and flipping coins (and a few other things too).\\

  \begin{exampleblock}{Example questions.}
    How exactly unsafe is playing russian roulette?
    
    I throw a fair $6$-sided die a million times. Can I say anything interesting \emph{a priori}:
    \begin{itemize}
    \item About the outcome of the $10000$th throw?
    \item About the number of times i'll get a $6$?
    \item About the average of all numbers I got?
    \end{itemize}
  \end{exampleblock}

  
\end{frame}


\begin{frame}
  \frametitle{Why a theory of probabilities?}
  
    \begin{itemize}
    \item Modelling \emph{Random} phenomena or \emph{trials}: when `repeated' trials yield different outcomes, \emph{i.e.} the outcome \emph{fluctuates}.
    \item Typical examples from before: throwing a die, flipping a coin.
    \item But not only: running a processor at a high speed (may crash or not), having $n$ radioactive element disintegrate in a given amount of time, transmitting a message through a faulty device...
    \item Draw conclusions, make prediction about outcomes and how much they can \emph{fluctuate}.
    \end{itemize}
    
\end{frame}

\begin{frame}
  \frametitle{Why a theory of probabilities (contd)?}
  Which of the following choice would you say is more interesting:
  
  \begin{exampleblock}{Choice 1.}
    I give you 1500 euros.
  \end{exampleblock}

  \begin{exampleblock}{Choice 2.}
    We play a game $1000$ times a row, where each of the 1000 plays consists in the following steps:
    \begin{enumerate}
    \item You roll a $6$-sided die over a over, until you get a $1$, $2$, $3$ or a $4$.
    \item You give me $1$ euro, and I give you $4$ euros for every time you re-rolled the die ($4 \times (n-1)$ for $n$ rolls).
    \end{enumerate}
  \end{exampleblock}

  \alert{No absolute answer, but after the course you'll know how to answer this for a specific meaning of `interesting'.}

\end{frame}

\begin{frame}{Uncertainty and incomplete knowledge.}

  \begin{itemize}
  \item Probabilities as a mean to measure \emph{degrees} of uncertainty.
  \item Guide rational decisions under uncertainty (work of Von Neunmann, Nash, \dots).
  \item No need to even \emph{believe} in non-determinism:
  \item probabilities as a view of the mind abstracting over unknown/too numerous/unmeasurable factors influencing a deterministic outcome.
  \item getting a simple and manageable approximation, allowing to draw conclusions and make predictions.
  \end{itemize}
  
\end{frame}

\begin{frame}
  \frametitle{Probabilities and language.}
  For a thorough treatment of this topic see [Manning \& Sch\"utze, \emph{Foundations of Natural Language Processing.}]. Here are a few points: 

  \begin{block}{Different possible expressions are not equally likely}
    Language use is best described accounting for the \emph{likelihood} of people to chose an expression over another.
  \end{block}

  \begin{exampleblock}{Convention}
    \begin{itemize}
    \item[a)] I did not shoot the deputy.
    \item[b)] I did not fire my gun on the deputy and hit him with a bullet.
    \end{itemize}
  \end{exampleblock}
\end{frame}

\begin{frame}
  \frametitle{Probabilities and language (contd).}
  
  \begin{block}{Degrees vs. categorical judgements.}
    In some cases, binary linguistics judgmnents (veracity, grammaticality) are hard to carry, and easier to model on a continuous scale.
    \begin{itemize}
    \item what are truth conditions for adjectives like \emph{bald} or \emph{red}?
    \end{itemize}
  \end{block}
\end{frame}

\begin{frame}
  \frametitle{Probabilities and language (contd).}
  \begin{block}{Ambiguity.}
    Ambiguity is present at every level of language, so that systematically asking for disambiguisation, or always entertaining all options would not be manageable.
  \end{block}
\end{frame}

\begin{frame}
  \frametitle{A technical aspect.}
             {\bf \emph{continuous} models offered by probability theory have exploitable mathematical property that discrete models lack: } they have enough regularity to allow various optimization algorithms to be used to find optimal parameters.
\end{frame}

\begin{frame}{Mathematical Foundations.}
  In this course, we will present the {\bf mathematical foundations} of probability theory. Means a formal language, and intuitive axioms such that:
 \begin{itemize}
  \item Real-life situations (such as previously evocated) might be abstractly described in the formal language.
  %\item Axioms are intuitively sound. %w.r.t. any real life situations that instanciate them.
  \item Pure mathematical reasoning might be used to draw deductions from a given situation's model
  \item[\alert{$\rightarrow$}] \alert{Provides grounded notion of valid reasoning, prevent following misleading intutions, and abstracts from customary details}.
  \item Conclusions can be translated back into concrete statements about the situation at hand.
  \end{itemize}
\end{frame}

\begin{frame}{Approach.}
  \begin{itemize}
  \item We want to talk about outcomes of random experiments. First need: a way to talk about different outcomes.
  \item Several routes possible. Most modern mathematiciens (and this course) follow the work of Kolmogorov:
  \item[$\rightarrow$]{\bf Root everything in set theory.}
  \end{itemize}
  
\end{frame}

\begin{frame}{Outcomes.}
  
  \begin{block}{Sample space}
    The collection of all possible outcomes of a random experiment is simply represented as a set. This set is called the \emph{sample space} or \emph{universe}. Most often denoted as $\Omega$.
  \end{block}

  For now we will always work with \emph{at most countable} sample spaces. {\color{blue} it means that $\Omega$ is either finite, or countably infinite (its elements can be numeroted $\Omega = \{ \omega_1, \dots, \omega_n, \dots\}$).}
\end{frame}

\begin{frame}{Examples}
  \begin{exampleblock}{$6$-sided die}
    The possible outcomes of rolling a six-sided die can be represented by the sample space $\Omega_{d}= \{ 1, 2, 3, 4, 5, 6 \}$.
  \end{exampleblock}

  \begin{exampleblock}{Two $6$-sided dice}
    The possible outcomes of rolling two six-sided dice can be represented by the sample space $\Omega_{dd}= \Omega_{d}\times \Omega_{d}$.
    \[ \Omega_{dd} = \left\{ \begin{array}{llllll}
      (1, 1), &(1, 2), &(1, 3), &(1, 4), &(1, 5), &(1, 6),\\
      (2, 1), &(2, 2), &(2, 3), &(2, 4), &(2, 5), &(2, 6),\\
      (3, 1), &(3, 2), &(3, 3), &(3, 4), &(3, 5), &(3, 6),\\
      (4, 1), &(4, 2), &(4, 3), &(4, 4), &(4, 5), &(4, 6),\\
      (5, 1), &(5, 2), &(5, 3), &(5, 4), &(5, 5), &(5, 6),\\
      (6, 1), &(6, 2), &(6, 3), &(6, 4), &(6, 5), &(6, 6)\\
      \end{array}
      \right \}.\]
  \end{exampleblock}

\end{frame}

\begin{frame}{Events}
  Talking about single outcomes only (as in ``I rolled a six") is too restricted. We need logically structured statements (``I rolled a one {\bf or} a two", `` I \textbf{did not} roll a five''). Set theory already provides the necessary tools:

  \begin{block}{Event}
    Let $\Omega$ be a sample space. An event $e$ (over $\omega$) is a part of $\Omega$. For an event $e \subseteq \Omega$, an outcome $\omega \in \Omega$ \emph{realises} event $e$ iff $\omega \in e$.
  \end{block}
\end{frame}

\begin{frame}{Example}
  \begin{exampleblock}{Two $6$-sided dice}
    \begin{itemize}
    \item[$e1$] ``the sum of the two rolls is 11'' $\rightarrow$ $e_{1} = \{ (5, 6), (6, 5) \}$.
    \item[$e2$] ``the two rolls are equal and both are even'' $\rightarrow$ $e_{2} = \{(2,2), (4,4), (6,6)\}$.
    \item[$e3$] ``the first dice's roll is 1'' $\rightarrow$ $e_{3} = \{(1,1), (1,2), (1,3), (1,4), (1,5), (1,6)\}$.
    \end{itemize}

    \[ \Omega_{dd} = \left\{ \begin{array}{llllll}
      (1, 1), &(1, 2), &(1, 3), &(1, 4), &(1, 5), &(1, 6),\\
      (2, 1), &(2, 2), &(2, 3), &(2, 4), &(2, 5), &(2, 6),\\
      (3, 1), &(3, 2), &(3, 3), &(3, 4), &(3, 5), &(3, 6),\\
      (4, 1), &(4, 2), &(4, 3), &(4, 4), &(4, 5), &(4, 6),\\
      (5, 1), &(5, 2), &(5, 3), &(5, 4), &(5, 5), &(5, 6),\\
      (6, 1), &(6, 2), &(6, 3), &(6, 4), &(6, 5), &(6, 6)\\
      \end{array}
      \right \}.\]
    
  \end{exampleblock}
\end{frame}

\begin{frame}{Events and propositions.}

  \begin{block}{Set theoretic manipulations.}
    There is an intuitive correspondence between logical operations on propositions and set-theoretic operations on events that encode these propositions:
    \begin{center}
      \begin{tabular}{|l|l|}
        \hline
        Proposition & Event\\
        \hline
        $A$ or $B$ & $e_{A} \cup e_{B}$\\
        \hline
        $A$ and $B$ & $e_{A} \cap e_{B}$\\
        \hline
        not $A$ & $\Omega \setminus A$\\
        \hline
        $\top$ (tautological statement) & $\Omega$\\
        \hline
        $\bot$ (impossible statement) & $\emptyset$\\
        \hline        
      \end{tabular}
    \end{center}

  \end{block}
  
\end{frame}

\begin{frame}
  \frametitle{Probability: some basic intuitions.}
  So far we described \emph{possible} events. Nothing yet about \emph{probabilities}. What are reasonable axioms?

  \begin{itemize}
  \item Being more ore less \emph{probable} is a property of \emph{events}.
  \item We can always compare the probability of two events. It is technically convenient to realise this by having probabilities as \emph{real numbers}.
  \item A certain event (\emph{e.g.} $\Omega$) has maximal probability. By convention, one assumes it's $1$.
  \item An impossible event (\emph{e.g.} $\emptyset$) has minimal probability. By convention one assumes it's $0$.
  \item[$\Rightarrow$] {\bf probability measure: function $p$ assigning any event $e$ a real number $p(e) \in [0, 1]$.}
  \end{itemize}
\end{frame}

\begin{frame}{Probability: principle of insufficient reason.}
    \begin{itemize}
    \item Throwing a fair $6$-sided die again.
    \item The die is fair. All outcomes should be equally likely.
    \item [$\Rightarrow$] assign same probability to all (elementary) \emph{outcomes}.
    \item By extension, two events realized by the same number of outcomes should have same probability too.
    \item {\color{darkgreen} \emph{e.g.} Rolling a number greater than $3$ vs rolling a number less than $3$.}
    \item  Probability over all \emph{events}?
    \end{itemize}

    Assuming $\Omega$ finite:
    \[ p(e) = \frac{|e|}{|\Omega|} \] 
\end{frame}

\begin{frame}{Examples}
  \begin{textblock*}{100pt}(200pt,50pt)
     $p(e) = \frac{|e|}{|\Omega|}$
  \end{textblock*}
  \begin{exampleblock}{$1$ die.}
    $\Omega_{d}= \{ 1, 2, 3, 4, 5, 6 \}$, $|\Omega_d| = 6$.
    \begin{itemize}
    \item $e_1: \{1\}$ -- rolling a one. $|e_1| = 1$, $p(e_1) = \frac{1}{6}$.
    \item $e_2$: $\{2\}$ -- rolling a $2$, $e_3: \{3\}$ rolling a $3$. $p(e_2) = p(e_3) = p(e_1) = \frac{1}{6}$ 
    \item $e_{\le 3}: e_1 \cup e_2 \cup e_3 = \{1,2,3\}$ -- rolling a number less or equal to $3$. $|e_{\le 3}| = 3$ $p(e_{\le 3}) = \frac{1}{2}$.
    \item Notice that $p(e_{\le 3}) = p(e_1) + p(e_2) + p(e_3)$.
    \item $\Omega_d$ -- rolling anything. $p(\Omega_d) = 1$.
    \item $\emptyset$ -- rolling nothing. $|\emptyset| = 0$, $p(\emptyset) = 0$.
    \end{itemize}
  \end{exampleblock}
\end{frame}

\begin{frame}{Examples}
  \begin{exampleblock}{$2$ dice.}
    \[ \Omega_{dd} = \left\{ \begin{array}{llllll}
      (1, 1), &(1, 2), &(1, 3), &(1, 4), &(1, 5), &(1, 6),\\
      (2, 1), &(2, 2), &(2, 3), &(2, 4), &(2, 5), &(2, 6),\\
      (3, 1), &(3, 2), &(3, 3), &(3, 4), &(3, 5), &(3, 6),\\
      (4, 1), &(4, 2), &(4, 3), &(4, 4), &(4, 5), &(4, 6),\\
      (5, 1), &(5, 2), &(5, 3), &(5, 4), &(5, 5), &(5, 6),\\
      (6, 1), &(6, 2), &(6, 3), &(6, 4), &(6, 5), &(6, 6)\\
      \end{array}
    \right \}, |\Omega_{dd}| = 36.\]
    \begin{itemize}
    \item $e'_1: \{ (1, 1), (1, 2), (1, 3), (1, 4), (1, 5), (1, 6)\}$ -- first dice rolling a $1$. $|e'_1| = 6$, $p(e'_1) = \frac{6}{36} = \frac{1}{6}$.
    \item Same result despite two different universes, phew...
    \end{itemize}
  \end{exampleblock}  
\end{frame}

\begin{frame}{Additivity.}
  \begin{itemize}
  \item What if elementary outcomes are {\bf not} all as likely?
  \item Which property should be kept?
  \end{itemize}

  \begin{block}{Equiprobable case}
    If $\Omega$ is finite and $p(e) = \frac{|e|}{|\Omega|}$, $e_a$ and $e_b$ are {\bf disjoint} events then
    \[p(e_a \cup e_b) = p(e_a) + p(e_b)\] (exercise).
  \end{block}

  \begin{itemize}
  \item This property is called additivity.
  \item We should keep that. Let's see why.
  \end{itemize}
  
\end{frame}

\begin{frame}{Additiviy and the frequentist interpretation.}
Several interpretations of probabilities. One of them is the \emph{frequentist interpretation:}
 
  \begin{itemize}
  \item If we repeat a given random experiment a sufficient number of times, the frequency of a occurence of a given event is its probability.
  \item {\color{darkgreen}  That the probability of obtaining $6$ when rolling a $6$-sided die is $1/6$ `means' that after sufficiently many trials we'll always have obtained $6$ in about $1/6$ of the total number of rolls.}
  \end{itemize}

 Not a mathematical definition, and complicated to work out as such (\emph{e.g.}, no formal definition so far of repeated trials). but help fix intuitions.
\end{frame}

\begin{frame}
  \frametitle{Additivity:}
  Frequentist interpretation $\sim$ probability = limit frequency of occurence of a given event.

  \begin{exampleblock}{Intuitition}
    \begin{itemize}
    \item Repeat an experiment $n$ times, \emph{e.g.} rolling a die.
    \item Consider \emph{disjoint} events $e_A$: rolling a $1$ and $e_B$: rolling a $6$.
    \item $n_{e_A}$ number of times $e_A$ is realised, $n_{e_B}$ number of times $e_b$ is realised.
    \item Number of times $e_A \cup e_B$ is realised: $n_{e_A} + n_{e_B}$. Frequency of occurence: $f(e_A \cup e_B) = \frac{n_{e_A}}{n} + \frac{n_{e_B}}{n} = f(e_A) + f(e_B) \sim \frac{1}{6} + \frac{1}{6} = \frac{1}{3}$ for $n$ large.
    \item[$\Rightarrow$] \alert{Under the frequentist interpretation, if $A$ and $B$ are disjoint events we must have $p(e_A \cup e_B) = p(e_A) + p(e_B)$.}
    \end{itemize}
  \end{exampleblock}
\end{frame}

\begin{frame}{Additivity (contd).}
  \begin{itemize}
  \item Other arguments can be made for other interpretations of probabilities.
  \item Frenquentist interpretation is not a definition. Instead take additivity as an axiom, then develop link between frequency and probabilities as a theorem (SPOILER: law of large numbers).
  \item For techincal reasons, need a more powerful version of additivity: $\sigma$-additivity -- next slide.
  \end{itemize}
\end{frame}


\begin{frame}{Axioms of probability}
  Let $\Omega$ be a sample space. A probability measure over $\Omega$ is a function $p: \wp(\Omega) \mapsto [0,1]$ with the two following properties:

  \begin{block}{Total mass.}
    $p(\Omega) = 1$
  \end{block}

  \begin{block}{$\sigma$-additivity}
    For any sequence of {\bf pairwise disjoint} events $\{e_1, \dots, e_n, \dots\}$ ($i \neq j \rightarrow e_i \cap e_j = \emptyset$),
    \[p(\bigcup_{i \in \mathbb{N}} e_i) = \sum^{\infty}_{1} p(e_i) \]
    \begin{itemize}
    \item Implies (binary) additivity (Exercise).
    \end{itemize}
  \end{block}
\end{frame}

\begin{frame}{Probabilities of individual outcomes determine the measure.}
  \begin{itemize}
  \item Remember that we supposed the sample space to be at most countable.
  \item Means we can write $\Omega = \{ w_1, \dots, w_n, \dots\}$.
  \end{itemize}

  {\bf Any sequence of positive real numbers $p_1, \dots, p_n, \dots$ such that $\sum^{\infty}_{i=1} p_i = 1$ uniquely determine a probability measure.}
\end{frame}

\begin{frame}
  \frametitle{Some properties.}
  \begin{enumerate}
  \item $p(e) \ge 0$ (positivity, by def.)
  \item $p(\emptyset) = 0$ (the impossible event has probability $0$). Follows from the disjoint union $\Omega = \Omega \cup \emptyset$ and additivity.
  \item $e_1 \subseteq e_2 \rightarrow p(e_1) \le p(e_2)$ (exercise, hint: $e_2 = e_1 \cup (e_2 \setminus e_1)$).
  \item $p(\neg e_1) = 1 - p(e_1)$ (probability of the negated event) (exercise).
  \end{enumerate}
\end{frame}

\begin{frame}
  \frametitle{Law of total probability.}
  \[ p(e_1 \cup e_2) = p(e_1) + p(e_2) - p(e_1 \cap e_2)\]

  \begin{block}<2->{proof.}
    \begin{itemize}
    \item<3-> $e_1 = (e_1 \cap e_2) \cup (e_1 \setminus e_2)$ (disjoint union).
    \item<4-> $p(e_1 \setminus e_2) = p(e_1) - p(e_1 \cap e_2)$ (additivity).
    \item<5-> Similarily $p(e_2 \setminus e_1) = p(e_2) - p(e_1 \cap e_2)$.
    \item<6-> $e_1 \cup e_2 = (e_1 \cap e_2) \cup (e_1 \setminus e_2) \cup (e_2 \setminus e_1)$ (disjoint union).
    \item<7-> $p(e_1 \cup e_2) = p(e_1 \cap e_2) + p(e_1 \setminus e_2) + p(e_2 \setminus e_1)$ (additivity).
    \item<8-> $p(e_1 \cup e_2) = p(e_1 \cap e_2) + p(e_1) - p(e_1 \cap e_2) + p(e_2) - p(e1 \cap e_2) = p(e_1) + p(e_2) - p(e_1 \cap e_2)$.
    \end{itemize}
  \end{block}
\end{frame}
  
\begin{frame}{Exercice}
  \begin{exampleblock}{Unfair die.}
    We want to model an unfair $6$-sided die such that the probability of rolling a $i$ is $i$ times the probability of rolling a $1$.
    \begin{itemize}
    \item Give such a probability measure over the sample space $\{ 1,2,3,4,5,6 \}$.
    \item How many such probability measure does it exist?
    \item Is it more likely to get a number less than or equal to $3$, or to get a $6$?
    \item Same questions for a $n$-sided dice (give a necessary and sufficient condition on $n$ for the last one).
    \end{itemize}
  \end{exampleblock}
  
\end{frame}

\end{document}
