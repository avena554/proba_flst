\documentclass{beamer}

%\usepackage{listings}
%\usepackage[francais]{babel}
\usepackage[T1]{fontenc}
\usepackage[utf8]{inputenc}
%\usepackage{MyriadPro}
\usepackage{cabin}
\usepackage{graphicx}
\usepackage{array}
\usepackage{tikz}
\usetikzlibrary{positioning, backgrounds, shapes, chains, decorations.pathmorphing}
\usepackage{amsmath,amsthm,amssymb}  
\usepackage{stmaryrd}
%\usepackage{mdsymbol}
\usepackage{MnSymbol}
\usepackage{xcolor}
\usepackage{verbatim}
\usepackage{array}
%\usepackage{csquotes}


\useoutertheme{infolines}

\newcommand{\hidden}[1]{}

%colors
\definecolor{darkgreen}{rgb}{0,0.5,0}
\usebeamercolor{block title}
\definecolor{beamerblue}{named}{fg}
\usebeamercolor{alert block title}
\definecolor{beamealert}{named}{fg}

\renewcommand{\colon}{\!:\!}


\newcommand\paraitem{%
 \quad
 \makebox[\labelwidth][r]{%
 \makelabel{%
 \usebeamertemplate{itemize \beameritemnestingprefix item}}}\hskip\labelsep}

\newcommand{\mmid}{\mathbin{{\mid}{\mid}}}

\begin{document}

\title{Probability Theory -- Introduction} 
\author{Antoine Venant}
%\institute{UDS COLI}
\date{\today}
\maketitle

\begin{frame}
  \frametitle{What is probability theory?}

  Probability theory is the branch of mathematics that deals with throwing dice and flipping coins (and a few other things too).\\

  \begin{exampleblock}{Example}
    I throw a fair $6$-sided die a million times. Can I say anything interesting \emph{a priori}:
    \begin{itemize}
    \item About the outcome of the $10000$th throw?
    \item About the number of times i'll get a $6$?
    \item About the average of all numbers I got?
    \end{itemize}
  \end{exampleblock}

  
\end{frame}


\begin{frame}
  \frametitle{Why a theory of probabilities?}
  
  \begin{block}{Nondeterminism}
    \begin{itemize}
    \item Modelling \emph{Random} phenomena or \emph{trials}: when `repeated' trials yield different outcomes, \emph{i.e.} the outcome \emph{fluctuates}.
    \item Typical examples from before: throwing a die, flipping a coin.
    \item But not only: playing roulette, running a processor at a certain speed (may crash or not), having $n$ radioactive element disintegrate in a given amount of time, transmitting a message through a faulty device...
    \item Draw conclusions, make prediction about outcomes and how much they can \emph{fluctuate}.
    \end{itemize}
  \end{block}  
\end{frame}

\begin{frame}
  \frametitle{Why a theory of probabilities (contd)?}
  Which of the following choice would you say is more interesting:
  
  \begin{exampleblock}{Choice 1.}
    I give you 1500 euros.
  \end{exampleblock}

  \begin{exampleblock}{Choice 2.}
    We play a game $1000$ times a row, where each of the 1000 plays consists in the following steps:
    \begin{enumerate}
    \item You roll a $6$-sided die over a over, until you get a $1$, $2$, $3$ or a $4$.
    \item You give me $1$ euro, and I give you $4$ euros for every time you re-rolled the die ($4 \times (n-1)$ for $n$ rolls).
    \end{enumerate}
  \end{exampleblock}

  \alert{No absolute answer, but after the course you'll know how to answer this for a specific meaning of `interesting'.}

\end{frame}

\begin{frame}
  
  \begin{block}{Uncertainty and incomplete knowledge.}
    \begin{itemize}
    \item Probabilities as a mean to measure \emph{degrees} of uncertainty.
    \item Guide rational decisions under uncertainty (work of Von Neunmann, Nash, \dots).
    \item No need to even \emph{believe} in non-determinism:
    \item probabilities as a view of the mind abstracting over unknown/too numerous/unmeasurable factors influencing a deterministic outcome.
    \item getting a simple and manageable approximation, allowing to draw conclusions and make predictions.
    \end{itemize}
  \end{block}
 
\end{frame}

\begin{frame}
  \frametitle{Probabilities and language.}
  For a thorough treatment of this topic see [Manning \& Sch\"utze, \emph{Foundations of Natural Language Processing.}]. Here are a few points: 

  \begin{block}{Different possible expressions are not equally likely}
    Language use is best described accounting for the \emph{likelihood} of people to chose an expression over another.
  \end{block}

  \begin{exampleblock}{Convention}
    \begin{itemize}
    \item[a)] I did not shoot the deputy.
    \item[b)] I did not fire my gun on the deputy and hit him with a bullet.
    \end{itemize}
  \end{exampleblock}
\end{frame}

\begin{frame}
  \frametitle{Probabilities and language (contd).}
  
  \begin{block}{Degrees vs. categorical judgements.}
    In some cases, binary linguistics judgmnents (veracity, grammaticality) are hard to carry, and easier to model on a continuous scale.
    \begin{itemize}
    \item what are truth conditions for adjectives like \emph{bald} or \emph{red}?
    \end{itemize}
  \end{block}
\end{frame}

\begin{frame}
  \frametitle{Probabilities and language (contd).}
  \begin{block}{Ambiguity.}
    Ambiguity is present at every level of language, so that systematically asking for disambiguisation, or always entertaining all options would not be manageable.
  \end{block}
\end{frame}

\begin{frame}
  \frametitle{A technical aspect.}
             {\bf \emph{continuous} models offered by probability theory have exploitable mathematical property that discrete models lack: } they have enough regularity to allow various optimization algorithms to be used to find optimal parameters.
\end{frame}

\begin{frame}{Mathematical Foundations.}
  In this course, we will present the {\bf mathematical foundations} of probability theory. Means a formal language, and axioms such that:
 \begin{itemize}
  \item Real-life situations (such as previously evocated) might be abstractly described in the formal language.
  \item Axioms are intuitively sound w.r.t. any real life situations that instanciate them.
  \item Pure mathematical reasoning might be used to draw deductions from a given situation's model
    \begin{itemize}
    \item[\alert{$\rightarrow$}] \alert{Provides grounded notion of valid reasoning, prevent following misleading intutions, and abstracts from customary details}.
    \end{itemize}
  \item Conclusions can be translated back into concrete statements about the situation at hand.
  \end{itemize}
\end{frame}

\begin{frame}{Approach.}
  \begin{itemize}
  \item We want to talk about outcomes of random experiments. First need: a way to talk about different outcomes.
  \item Several routes possible. Most modern mathematiciens (and this course) follow the work of Kolmogorov:
  \item[$\rightarrow$]{\bf Root everything in set theory.}
  \end{itemize}
  
\end{frame}

\begin{frame}{Outcomes.}
  
  \begin{block}{Sample space}
    The collection of all possible outcomes of a random experiment is simply represented as a set. This set is called the \emph{sample space} or \emph{universe}. Most often denoted as $\Omega$.
  \end{block}

  For now we will always work with \emph{at most countable} sample spaces. {\color{blue} it means that $\Omega$ is either finite, or countably infinite (its elements can be numeroted $\Omega = \{ \omega_1, \dots, \omega_n, \dots\}$).}
\end{frame}

\begin{frame}{Examples}
  \begin{exampleblock}{$6$-sided die}
    The possible outcomes of rolling a six-sided die can be represented by the sample space $\Omega_{d}= \{ 1, 2, 3, 4, 5, 6 \}$.
  \end{exampleblock}

  \begin{exampleblock}{Two $6$-sided dice}
    The possible outcomes of rolling two six-sided dice can be represented by the sample space $\Omega_{dd}= \Omega_{d}\times \Omega_{d}$.
    \[ \Omega_{dd} = \left\{ \begin{array}{llllll}
      (1, 1), &(1, 2), &(1, 3), &(1, 4), &(1, 5), &(1, 6),\\
      (2, 1), &(2, 2), &(2, 3), &(2, 4), &(2, 5), &(2, 6),\\
      (3, 1), &(3, 2), &(3, 3), &(3, 4), &(3, 5), &(3, 6),\\
      (4, 1), &(4, 2), &(4, 3), &(4, 4), &(4, 5), &(4, 6),\\
      (5, 1), &(5, 2), &(5, 3), &(5, 4), &(5, 5), &(5, 6),\\
      (6, 1), &(6, 2), &(6, 3), &(6, 4), &(6, 5), &(6, 6)\\
      \end{array}
      \right \}.\]
  \end{exampleblock}

\end{frame}

\begin{frame}{Events}
  Talking about single outcomes only (as in ``I rolled a six") is too restricted. We need logically structured statements (``I rolled a one {\bf or} a two", `` I \textbf{did not} roll a five''). Set theory already provides the necessary tools:

  \begin{block}{Event}
    Let $\Omega$ be a sample space. An event $e$ (over $\omega$) is a part of $\Omega$. For an event $e \subseteq \Omega$, an outcome $\omega \in \Omega$ \emph{realises} event $e$ iff $\omega \in e$.
  \end{block}
\end{frame}

\begin{frame}{Example}
  \begin{exampleblock}{Two $6$-sided dice}
    \begin{itemize}
    \item[$e1$] ``the sum of the two rolls is 11'' $\rightarrow$ $e_{1} = \{ (5, 6), (6, 5) \}$.
    \item[$e2$] ``the two rolls are equal and both are even'' $\rightarrow$ $e_{2} = \{(2,2), (4,4), (6,6)\}$.
    \item[$e3$] ``the first dice's roll is 1'' $\rightarrow$ $e_{3} = \{(1,1), (1,2), (1,3), (1,4), (1,5), (1,6)\}$.
    \end{itemize}

    \[ \Omega_{dd} = \left\{ \begin{array}{llllll}
      (1, 1), &(1, 2), &(1, 3), &(1, 4), &(1, 5), &(1, 6),\\
      (2, 1), &(2, 2), &(2, 3), &(2, 4), &(2, 5), &(2, 6),\\
      (3, 1), &(3, 2), &(3, 3), &(3, 4), &(3, 5), &(3, 6),\\
      (4, 1), &(4, 2), &(4, 3), &(4, 4), &(4, 5), &(4, 6),\\
      (5, 1), &(5, 2), &(5, 3), &(5, 4), &(5, 5), &(5, 6),\\
      (6, 1), &(6, 2), &(6, 3), &(6, 4), &(6, 5), &(6, 6)\\
      \end{array}
      \right \}.\]
    
  \end{exampleblock}
\end{frame}

\begin{frame}{Events and propositions.}

  \begin{block}{Set theoretic manipulations.}
    There is an intuitive correspondance between logical operations on propositions and set-theoretic operations on events that encode these propositions:
    \begin{center}
      \begin{tabular}{|l|l|}
        \hline
        Proposition & Event\\
        \hline
        $A$ or $B$ & $e_{A} \cup e_{B}$\\
        \hline
        $A$ and $B$ & $e_{A} \cap e_{B}$\\
        \hline
        not $A$ & $\Omega \setminus A$\\
        \hline
        $\top$ (tautological statement) & $\Omega$\\
        \hline
        $\bot$ (impossible statement) & $\emptyset$\\
        \hline        
      \end{tabular}
    \end{center}

  \end{block}
  
\end{frame}

\begin{frame}
  \frametitle{Axioms of probability.}
\end{frame}

\end{document}
