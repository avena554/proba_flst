
\documentclass{beamer}

%\usepackage{listings}
%\usepackage[francais]{babel}
\usepackage[T1]{fontenc}
\usepackage[utf8]{inputenc}
%\usepackage{MyriadPro}
\usepackage{cabin}
\usepackage{graphicx}
\usepackage{array}
\usepackage{tikz}
\usetikzlibrary{positioning, backgrounds, shapes, chains, decorations.pathmorphing}

\usepackage{amsmath,amsthm,amssymb}  
\usepackage{stmaryrd}
%\usepackage{mdsymbol}
\usepackage{MnSymbol}
\usepackage{xcolor}
\usepackage{verbatim}
\usepackage{array}
%\usepackage{csquotes}



\usepackage[absolute,overlay]{textpos}
%\usepackage[texcoord,
%grid,gridcolor=red!10,subgridcolor=green!10,gridunit=pt]
%{eso-pic}



\useoutertheme{infolines}

\newcommand{\hidden}[1]{}

%colors
\definecolor{darkgreen}{rgb}{0,0.5,0}
\usebeamercolor{block title}
\definecolor{beamerblue}{named}{fg}
\usebeamercolor{alert block title}
\definecolor{beamealert}{named}{fg}

\renewcommand{\colon}{\!:\!}


\newcommand\paraitem{%
 \quad
 \makebox[\labelwidth][r]{%
 \makelabel{%
 \usebeamertemplate{itemize \beameritemnestingprefix item}}}\hskip\labelsep}

\newcommand{\mmid}{\mathbin{{\mid}{\mid}}}

\begin{document}

\title{Matrices.} 
\author{Antoine Venant}
%\institute{UDS COLI}
\date{\today}
\maketitle


\begin{frame}{Previously:}
  \begin{itemize}
  \item Linear map between vector spaces $V$ and $W$.
  \item \emph{Determined by the image of the vectors of a basis of $V$}.
  \end{itemize}
\end{frame}

\begin{frame}{Today:}
  \begin{itemize}
  \item Use this property to define a representation suitable for calculus
  \item Matrices.
  \item Matrices are representation of linear maps w.r.t. a given basis $V$ and a given basis of $W$.
  \end{itemize}
\end{frame}

\hidden{
\begin{exampleblock}{Example:}
  \begin{center}
      \input{automorphism}      
    \end{center}
\end{exampleblock}
}

\begin{frame}
  \frametitle{Example}
  \begin{exampleblock}{A linear map}
    \[f: \begin{aligned} \mathbb{R}^2 &\mapsto \mathbb{R}^2\\ (x,y) &\mapsto (x - y, y - x) \end{aligned}\]

    Two basis of $\mathbb{R}^2 :$
    \[\begin{aligned}
    &\mathcal{E} = \langle e_1, e_2 \rangle = \langle (1,0), (0,1) \rangle \\ 
    &\mathcal{B} = \langle \alpha, \beta \rangle = \langle (1,1), (-1, 1) \rangle\\
    \end{aligned}
    \]
    \[
    \begin{aligned}
      &f(e_1) = (1,-1) = {\color{blue} \langle 1, -1 \rangle_{\mathcal{E}}} = {\color{darkgreen} \langle 0, -1\rangle_{\mathcal{B}}}\\
      &f(e_2) = (-1, 1) = {\color{blue} \langle -1, 1 \rangle_{\mathcal{E}}} = {\color{darkgreen} \langle 0, 1 \rangle_{\mathcal{B}}} \\      
    \end{aligned}
    \]


    \[f:
    \begin{array}{|ccc|}
      f({\color{red}e_1}) & f({\color{red}e_2}) & \\
      \color{blue}\phantom{-}1 & \color{blue} -1 & \color{blue} e_1\\
      \color{blue}-1 & \color{blue} \phantom{-}1 & \color{blue} e_2
    \end{array}_{{\color{red}\mathcal{E}}, {\color{blue}\mathcal{E}}}
    \equiv
    \begin{array}{|ccc|}
      f({\color{red}e_1}) & f({\color{red}e_2}) & \\
      \color{darkgreen}\phantom{-}0 & \color{darkgreen} \phantom{-}0 & \color{darkgreen} \alpha\\
      \color{darkgreen} -1 & \color{darkgreen}\phantom{-}1 & \color{darkgreen} \beta
    \end{array}_{{\color{red}\mathcal{E}}, {\color{darkgreen} \mathcal{B} }}
    \]

    
  \end{exampleblock}
\end{frame}


\begin{frame}
  {Matrix}
  Let $m, n \in \mathbb{N}$.
  \begin{block}{Definition}
    An $(n,m)$ matrix is a sequence of $n \times m$ real coefficients $a_{i,j}$ with $1 \le i \le n, 1 \le j \le m$. We can organize them visually in the following way:

    \[\begin{array}{|ccc|} a_{1,1} & \dots & a_{1,m}\\ \vdots & & \vdots\\ a_{n,1} & \dots & a_{n, m} \end{array}\]
  \end{block}
\end{frame}


\begin{frame}{Matrix $\Rightarrow$ linear map}
  \begin{block}{Definition}
    To any $n,m$ matrix $A = \langle a_{i,j} \rangle$ we associate an application $\hat A : \left \{ \begin{array}{l} \mathbb{R}^m \mapsto \mathbb{R}^n \\ X \rightarrow A \times X \end{array} \right .$ as follows:
    \[
    \begin{array}{|ccc|} a_{1,1} & \dots & a_{1,m}\\ \vdots & & \vdots\\ a_{n,1} & \dots & a_{n, m} \end{array} \times \left ( \begin{array}{c} x_1 \\ \vdots \\ x_m \end{array} \right )_{X} = \left ( \begin{array}{c} a_{1,1}x_1 + \dots + a_{1,m}x_m \\ \vdots \\ a_{n,1}x_1 + \dots + a_{n,m} x_m \end{array} \right )  
    \]  
  \end{block}

  \begin{block}{Proposition}
    The associated application $\hat A$ defined above is a linear map.
  \end{block}
\end{frame}

\begin{frame}{Example}
  \begin{exampleblock}{$2,2$ matrix}
    \[ \begin{array}{|cc|}
      \phantom{-}1 &  -1 \\
      -1 & \phantom{-}1 
    \end{array} \times \left ( \begin{array}{c} x \\ y \end{array} \right ) = \left ( \begin{array}{c} x - y \\ y - x \end{array} \right )  
    \]
  \end{exampleblock}
\end{frame}

\begin{frame}{Reminder: canonical basis}
  \begin{block}{Canonical basis of $\mathcal{R}^m$}
    Letting $e_i = \langle \overbrace{0,\dots,0}^{(i-1) \textnormal{ times}}, 1, \overbrace{0,\dots,0}^{(m-i) \textnormal{ times}} \rangle$ we have: \[\langle e_1, \dots, e_m \rangle \textnormal{ is the canonical basis of } \mathbb{R}^m.\]
  \end{block}
\end{frame}

\begin{frame}
  \frametitle{Linear map $\Rightarrow$ matrix}
  \begin{block}{Converse}
    Let $f : \mathbb{R}^m \rightarrow \mathbb{R}^n$ linear. There exists a unique matrix $A$ such that $\hat A = f$.
  \end{block}

  \begin{block}{proof}
    \begin{itemize}
    \item We know that $f$ is uniquely determined by the images of the canonical basis of $\mathbb{R}^m$: $\langle e_1, \dots, e_m \rangle$.
    \item by def, $f(e_1), \dots, f(e_m) \in \mathbb{R}^n$.
    \item So let $\langle a_{1,1}, \dots, a_{n,1} \rangle = f(e_1) \dots \langle a_{1, m} \dots a_{m, n} \rangle = f(e_m)$.
    \item Then $A = \begin{array}{|ccc|} a_{1,1} & \dots & a_{1,m}\\ \vdots & & \vdots\\ a_{n,1} & \dots & a_{n, m} \end{array}$ is such that $\hat A = f$:
    \item suffices to check $\hat A(e_1) = A \times e_1 = f(e_1) \dots \hat A(e_m) = A \times e_m = f(e_m)$.
    \end{itemize}
  \end{block}
  
\end{frame}




\end{document}
