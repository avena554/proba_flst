
\documentclass{beamer}

%\usepackage{listings}
%\usepackage[francais]{babel}
\usepackage[T1]{fontenc}
\usepackage[utf8]{inputenc}
%\usepackage{MyriadPro}
\usepackage{cabin}
\usepackage{graphicx}
\usepackage{array}
\usepackage{tikz}
\usetikzlibrary{positioning, backgrounds, shapes, chains, decorations.pathmorphing}

\usepackage{amsmath,amsthm,amssymb}  
\usepackage{stmaryrd}
%\usepackage{mdsymbol}
\usepackage{MnSymbol}
\usepackage{xcolor}
\usepackage{verbatim}
\usepackage{array}
%\usepackage{csquotes}



\usepackage[absolute,overlay]{textpos}
%\usepackage[texcoord,
%grid,gridcolor=red!10,subgridcolor=green!10,gridunit=pt]
%{eso-pic}



\useoutertheme{infolines}

\newcommand{\hidden}[1]{}

%colors
\definecolor{darkgreen}{rgb}{0,0.5,0}
\usebeamercolor{block title}
\definecolor{beamerblue}{named}{fg}
\usebeamercolor{alert block title}
\definecolor{beamealert}{named}{fg}

\renewcommand{\colon}{\!:\!}


\newcommand\paraitem{%
 \quad
 \makebox[\labelwidth][r]{%
 \makelabel{%
 \usebeamertemplate{itemize \beameritemnestingprefix item}}}\hskip\labelsep}

\newcommand{\mmid}{\mathbin{{\mid}{\mid}}}

\begin{document}

\title{Exercises} 
\author{Antoine Venant}
%\institute{UDS COLI}
\date{\today}
\maketitle

\begin{frame}{Canonical Basis.}
  Show that the canonical basis of $\mathbb{R}^n$ is indeed a basis.
  
  \begin{exampleblock}{Reminder:}
    The canonical basis of $\mathbb{R}^n$ is $\mathcal{E}^n = \langle e^{(1)} \dots e^{(n)}\rangle$ with \[e^{(i)} = \langle \overbrace{0,\dots,0}^{(i-1) \textnormal{ times}}, 1, \overbrace{0,\dots,0}^{(n-i) \textnormal{ times}} \rangle\].
  \end{exampleblock}
 
\end{frame}

\begin{frame}{Direct sum.}
  \begin{block}{Definition.}
    Let $V$ be a vector space. If $W_1, W_2$ are vector subspaces such that $W_1 + W_2 = V$ and $W_1 \cap W_2 = \{0\}$, one says that $W_1$ and $W_2$ are in \emph{direct} sum and write \[V = W_1 \bigoplus W_2\]
  \end{block}

  Prove the following proposition:
  \begin{block}{Proposition}
    If $V = W_1 \bigoplus W_2$ then for any $v \in V$ there exists a \emph{unique} pair $w_1 \in W_1, w_2 \in W_2$ such that $v = w_1 + w_2$.
  \end{block}
\end{frame}



\begin{frame}{Exercise}
  One considers the vector space $\langle \mathbb{R}^{\mathbb{R}}, {+}, {\cdot}\rangle$ of real-valued functions of one real variable, where:
  \begin{itemize}
  \item $f+g$ is the function associating any $x \in \mathbb{R}$ with $[f+g](x) = f(x) + g(x)$.
  \item $\alpha \cdot f$ is the function associating any $x \in \mathbb{R}$ with $[\alpha \cdot f ](x) = \alpha \times f(x)$.
  \end{itemize}

  \begin{enumerate}
  \item Prove the above defined space is a vector space. In particular, what is the $0$ of this space?
  \end{enumerate}
  
  A function is \emph{even} iff for any $x \in \mathcal{R}$, $f(x) = f(-x)$. A function is \emph{odd} iff for any $x \in \mathcal{R}$, $f(x) = -f(-x)$. Let $E$ and $O$ respectively denote the sets of even and odd functions.
  \begin{enumerate}
  \item[3] Show that $E$ and $O$ are vector subspaces of $\mathcal{R}^{\mathcal{R}}$.
  \item[4] Show that $\mathbb{R}^{\mathbb{R}} = E \bigoplus O$.
  \item[6] Conclude that any function is {\textbf{uniquely}} decomposed as the sum of an even function and an odd function.
  \end{enumerate}  
\end{frame}

\begin{frame}{Exercise}
  \begin{enumerate}
  \item Let $cos^2$ and $sin^2$ respectively denote the functions $x \mapsto cos(x)^2$ and $x \mapsto sin(x)^2$. Consider the set $W = \{\lambda \cos^2 + \mu sin^2 \mid \lambda, \mu \in \mathbb{R} \}$. Appeal to a result of the lecture to briefly justify that $W$ is a vector (sub)space.
  \item Show that $W$ has dimension $2$.
  \item Justify whether or not the following functions are in $W$:
    \begin{itemize}
    \item $x \mapsto 1$
    \item $x \mapsto 1+x^2$
    \item $x \mapsto cos(2x)$
    \end{itemize}
  \end{enumerate}
\end{frame}


\begin{frame}{Completion lemma}
  Let $V$ be a vector space and let $\mathcal{V} = \langle v_1, \dots, v_k \rangle$ be a linearly independant sequence of vectors of $V$.
  \begin{enumerate}
  \item Show that for any $w \notin \mathcal{L}(\mathcal{V})$, $\langle v_1, \dots, v_k, w \rangle$ is linearly independent.
  \end{enumerate}
  Let $ \mathcal{W} = \langle w_1, \dots, w_l \rangle$ be a sequence of vectors of $V$ such that $\langle v_1, \dots, v_k, w_1, \dots, w_l \rangle$ is a generator of  $W$.
  \begin{enumerate}
    \setcounter{enumi}{2}
  \item Show that if $\mathcal{V}$ is not a basis of $\mathbb{R}$ then there exists $i \in [| 1, l |]$ such that $w_i \notin \mathcal{L}(\mathcal{V})$.
  \item Prove the completion theorem (see below for a reminder).
  \end{enumerate}

  \begin{exampleblock}{Completion theorem.}
      If $\langle v_1, \dots, v_k \rangle$ is linearly independent and $\langle w_1, \dots, w_l \rangle$ is a sequence of vectors such that $\langle v_1, \dots, v_k, w_1, \dots, w_l \rangle$ is a generator of $V$, then there exists indices $i_1, \dots, i_m$ ($m \le l$) such that $\langle v_1, \dots, v_n, w_{i_1}, \dots, w_{i_m} \rangle$ is a basis of $V$.
  \end{exampleblock}
\end{frame}

\begin{frame}{Property of finite dimensional spaces.}
  \begin{exampleblock}{Vocabulary}
    We call a vector space \emph{finite dimensional} if it admits a finite\footnote{our definitions actually consider only finite basis, but can be extended to the infinite case.}.
  \end{exampleblock}

  Prove the following properties for $V$ a finite dimensional space of dimension $n$:
  \begin{block}{Propositions}
    \begin{itemize}
    \item If $\langle v_1, \dots, v_k \rangle$ is linearly independent then $k \le n$.
    \item If $\langle v_1, \dots, v_k \rangle$ is a generator of $V$ then $k \ge n$.
    \item If $\langle v_1, \dots, v_n \rangle$ is linearly independent then it is a basis.
    \item If $\langle v_1, \dots, v_n \rangle$ is a generator of $V$ then it is a basis.
    \end{itemize}
  \end{block}
\end{frame}


\begin{frame}
  \frametitle{Exchange lemma.}
  Prove the exchange lemma (reminder below).
  
  \begin{block}{Lemma}
    If $\langle v_1, \dots, v_n \rangle$ and $\langle w_1, \dots, w_m \rangle$ are two basis of $V$, then for any $i \in [|1, n|]$ there exist $j \in [|1, m|]$ such that
    \[\langle v_{1}, \dots, v_{i-1}, w_j, v_{i+1}, \dots v_n \rangle \textnormal{ is a basis of } V.\]
  \end{block}
  
\end{frame}

\begin{frame}
  \frametitle{Exercise}
  Show that the following system of linear equation as a unique solution:
  \[
  \begin{array}{l}
    x + y + z = 0\\
    x - z = 0 \\
    y - 2z = 0\\
  \end{array}
  \]

  Show that $\mathcal{B} = \langle (1/4, 1/4, 0), (1/4,0,1/4), (1/4, -1/4, -1/2) \rangle$ is a basis of $\mathbb{R}^3$.
 
\end{frame}

\begin{frame}
  \frametitle{Exercise}
  \begin{itemize}
  \item What is the representation of $(1,1,1)$ with respect to the basis $\mathcal{B} = \langle (1/4, 1/4, 0), (1/4,0,1/4), (1/4, -1/4, -1/2) \rangle$ is a basis of $\mathbb{R}^3$?
  \end{itemize}
\end{frame}

\begin{frame}{Excercise}
  Consider the following differential equation, where $a, b \in \mathbb{R}^*$ are functions:
  \[ af'(x) + bf(x) = 0.\]
  Show that the set of solutions to this equation is a vector space of dimension $1$.
\end{frame}




\end{document}
