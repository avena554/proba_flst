
\documentclass{beamer}

%\usepackage{listings}
%\usepackage[francais]{babel}
\usepackage[T1]{fontenc}
\usepackage[utf8]{inputenc}
%\usepackage{MyriadPro}
\usepackage{cabin}
\usepackage{graphicx}
\usepackage{array}
\usepackage{tikz}
\usetikzlibrary{positioning, backgrounds, shapes, chains, decorations.pathmorphing}

\usepackage{amsmath,amsthm,amssymb}  
\usepackage{stmaryrd}
%\usepackage{mdsymbol}
\usepackage{MnSymbol}
\usepackage{xcolor}
\usepackage{verbatim}
\usepackage{array}
\usepackage{multirow}
%\usepackage{csquotes}



\usepackage[absolute,overlay]{textpos}
%\usepackage[texcoord,
%grid,gridcolor=red!10,subgridcolor=green!10,gridunit=pt]
%{eso-pic}



\useoutertheme{infolines}

\newcommand{\hidden}[1]{}

%colors
\definecolor{darkgreen}{rgb}{0,0.5,0}
\usebeamercolor{block title}
\definecolor{beamerblue}{named}{fg}
\usebeamercolor{alert block title}
\definecolor{beamealert}{named}{fg}

\renewcommand{\colon}{\!:\!}


\newcommand\paraitem{%
 \quad
 \makebox[\labelwidth][r]{%
 \makelabel{%
 \usebeamertemplate{itemize \beameritemnestingprefix item}}}\hskip\labelsep}

\newcommand{\mmid}{\mathbin{{\mid}{\mid}}}

\begin{document}

\title{Summary and complements.} 
\author{Antoine Venant}
%\institute{UDS COLI}
\date{\today}

\maketitle


\section{Basis and dimension}

\begin{frame}{Basis}
  Let $V$ be a vector space.

  \begin{itemize}
  \item Basis: a sequence of vector $\mathcal{B} = \langle v_1 \dots v_n \rangle$ which is:
    \begin{itemize}
    \item A generator, \emph{i.e.} $\mathcal{L}(v_1, \dots, v_n ) = V$ \emph{i.e.} \[\forall v \in V\, \exists \lambda_1, \dots, \lambda_n \in \mathbb{R}\, v = \lambda_1 v_1 + \dots + \lambda_n v_n.\]
    \item linearly independent \emph{i.e.} \[\forall \lambda_1,\dots, \lambda_n \in \mathbb{R} \, \lambda_1 v_1 + \dots + \lambda_n v_n = 0 \Rightarrow \lambda_1 = \dots = \lambda_n = 0.\]
    \end{itemize}
  \item When $\mathcal{B}$ is a basis, any vector $v$ admits a {\bf unique} decomposition $v = \lambda_1 v_1 + \dots + \lambda_n v_n$. We write \[v = \langle v_1, \dots, v_n \rangle_{\mathcal{B}}.\]
  \end{itemize}
\end{frame}

\begin{frame}{The completion Theorem}
  \begin{block}{Theorem}
    If $\langle v_1, \dots, v_k \rangle$ is linearly independent and $\langle w_1, \dots, w_l \rangle$ is a sequence of vectors such that $\langle v_1, \dots, v_k, w_1, \dots, w_l \rangle$ is a generator of $V$, then there exists indices $i_1, \dots, i_m$ ($m \le l$) such that $\langle v_1, \dots, v_n, w_{i_1}, \dots, w_{i_m} \rangle$ is a basis of $V$.
  \end{block}

  \alert{``If one has a linear independent family and a generator family, one can complete the first using vectors of the second to obtain a basis.''}
\end{frame}

\begin{frame}{Important consequences.}
  \begin{itemize}
  \item Every basis have the same size -- the \emph{dimension} of the space (we write $dim(V)$).
  \end{itemize}
  Let $V$ be a vector space with $dim(V) = n$
  \begin{block}{Propositions}
    \begin{itemize}
    \item If $\langle v_1, \dots, v_k \rangle$ is linearly independent then $k \le n$.
    \item If $\langle v_1, \dots, v_k \rangle$ is a generator of $V$ then $k \ge n$.
    \item If $\langle v_1, \dots, v_n \rangle$ is linearly independent then it is a basis.
    \item If $\langle v_1, \dots, v_n \rangle$ is a generator of $V$ then it is a basis.
    \item If $W \subseteq V$ is a vector subspace of $V$, $dim(W) \le dim(V)$.
    \item If $W \subseteq V$ is a vector subspace of $V$, $dim(W) = n$ iff $W = V$.
    \end{itemize}
  \end{block}
\end{frame}


\section{The rank theorem.}

\begin{frame}{Kernel and Image}
  \begin{itemize}
  \item $V, W$ finite dimensional vector spaces.
  \item $f: V \mapsto W$ linear map.
  \end{itemize}

  \begin{block}{Definition}
    The image of $f$ is the set of images of all vectors of $V$:
    \[Im(f) = \{ f(x) \mid x \in V \} \]
  \end{block}

  \begin{block}{Definition}
    The kernel of $f$ is the set of vectors that are mapped to $0$.
    \[Ker(f) = \{ x \mid f(x) = 0\}\]
  \end{block}

  \begin{itemize}
  \item The {\textbf{rank}} of a $f$ is $rk(f)=dim(Im(f))$.
  \item A linear map is injective iff $Ker(f) = \{0\}$
  \end{itemize}  
\end{frame}

\begin{frame}
  \frametitle{The rank theorem}
  Let $V$ and $W$ be {\bf finite dimensional} vector spaces and $f: V \mapsto W$ be a linear map.

  \begin{block}{Theorem}
    \[dim(Ker(f)) + dim(Im(f)) = dim(V)\]
  \end{block}
\end{frame}

\begin{frame}
  \begin{block}{Corollary of the rank theorem.}
    \begin{itemize}
    \item $V, W$ be vector spaces such that $dim(V) = dim(W) = n$.
    \item $f: V \mapsto W$ linear map.
    \item $\mathcal{B} = \langle v_1, \dots, v_n\rangle$ basis of $V$.
    \end{itemize}
      The following assertions are all equivalent:
      \begin{itemize}
      \item $f$ is injective
      \item $f$ is surjective
      \item $f$ is an isomorphism
      \item $rk(f) = n$.
      \item $\langle f(v_1), \dots, f(v_n) \rangle$ basis of $Im(f)$.
      \end{itemize}
  \end{block}
\end{frame}

\begin{frame}
  \frametitle{proof}
  \[
  \begin{aligned}
   f \textnormal{ injective } &\textnormal{ iff } Ker(f) = \{0\}\\
    &\textnormal{ iff } rk(f) = dim(Im(f)) = dim(V) - dim(Ker(f)) = n - 0 = n\\
    &\textnormal{ iff } Im(f) = W\\
    &\textnormal{ iff } f \textnormal{ surjective (csq of the completion thm -- see previous slides)}.\\
   &\textnormal{ iff } f\textnormal{ isomorphism}.
  \end{aligned}
  \]
\end{frame}

\begin{frame}
  \frametitle{Proof of last equivalence:}
  Remain to show the equivalence $dim(Im(f)) = n$ with: $\langle f(v_1), \dots, f(v_n) \rangle$ basis of $Im(f)$:
  \begin{itemize}
  \item If $\langle f(v_1), \dots, f(v_n) \rangle$ basis of $Im(f)$ then $dim(Im(f)) = n$.
  \item Conversely, assume $dim(Im(f)) = n$. $\langle f(v_1), \dots, f(v_n) \rangle$ is always a generator of $Im(f)$: $w \in Im(f)$ iff there exists  $v \in V$ such that $w = f(v)$. We can decompose $v$ in $\mathcal{B}$ to get $ v = \lambda_1 v_1 +  \dots + \lambda_n$. Then by linearity $w = f(v) = \lambda_1 f(v_1) + \dots + \lambda_n f(v_n)$. So $w \in \mathcal{L}(\langle f(v_1), \dots, f(v_n) \rangle)$. We've already seen as consequence of the completion theorem that a generator with $n$ vectors of a dim $n$ space is a basis.
  \end{itemize}
\end{frame}



\section{Complement: canonical isomorphisms.}

\begin{frame}{Canonical homomorphism.}
  \begin{itemize}
  \item $V$ vector space and $\mathcal{B} = \langle v_1 \dots v_n \rangle$ basis of $V$. 
  \end{itemize}

  \begin{definition}[Interpretation and representation w.r.t $\mathcal{B}$:]
    \begin{itemize}
    \item $\langle \cdot \rangle_{\mathcal{B}}: \begin{cases} \mathbb{R}^n \mapsto V \\ (x_1, \dots, x_n) \mapsto \langle x_1,\dots, x_n \rangle_{\mathcal{B}} = x_1 v_1 + \dots + x_n v_n  \end{cases}$.
    \item $\cdot \restriction_{\mathcal{B}}: \begin{cases} V \mapsto \mathbb{R}^n \\ v \mapsto v \restriction_{\mathcal{B}} = (x_1, \dots, x_n) \textnormal{ such that } v = \langle x_1, \dots, x_n \rangle_{\mathcal{B}} \end{cases}$.
    \end{itemize}
  \end{definition}

  \begin{itemize}
  \item {\bf Because $\mathcal{B}$ is a basis} the second definition is valid.
  \item Also, both are linear maps.
  \item Both are isomorphisms (check any of the conditions in privous slide about isomorphisms).
  \item They are each other inverse:
    \[
   (\langle x_1, \dots, x_n\rangle_{\mathcal{B}})\restriction_{B} = (x_1, \dots, x_n)
    \textnormal{ and }\langle v \restriction_{\mathcal{B}} \rangle_{\mathcal{B}} = v
    \]
  \end{itemize}
\end{frame}


\section{Matrices}


\section{Matrices to linear maps and back.}

\begin{frame}{$(n,m)$ Matrices $\Leftrightarrow$ linear maps $\mathcal{R}^m \mapsto \mathcal{R}^n$}
  bla
\end{frame}








\end{document}
