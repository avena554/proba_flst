
\documentclass{beamer}

%\usepackage{listings}
%\usepackage[francais]{babel}
\usepackage[T1]{fontenc}
\usepackage[utf8]{inputenc}
%\usepackage{MyriadPro}
\usepackage{cabin}
\usepackage{graphicx}
\usepackage{array}
\usepackage{tikz}
\usetikzlibrary{positioning, backgrounds, shapes, chains, decorations.pathmorphing}

\usepackage{amsmath,amsthm,amssymb}  
\usepackage{stmaryrd}
%\usepackage{mdsymbol}
\usepackage{MnSymbol}
\usepackage{xcolor}
\usepackage{verbatim}
\usepackage{array}
%\usepackage{csquotes}



\usepackage[absolute,overlay]{textpos}
%\usepackage[texcoord,
%grid,gridcolor=red!10,subgridcolor=green!10,gridunit=pt]
%{eso-pic}



\useoutertheme{infolines}

\newcommand{\hidden}[1]{}

%colors
\definecolor{darkgreen}{rgb}{0,0.5,0}
\usebeamercolor{block title}
\definecolor{beamerblue}{named}{fg}
\usebeamercolor{alert block title}
\definecolor{beamealert}{named}{fg}

\renewcommand{\colon}{\!:\!}


\newcommand\paraitem{%
 \quad
 \makebox[\labelwidth][r]{%
 \makelabel{%
 \usebeamertemplate{itemize \beameritemnestingprefix item}}}\hskip\labelsep}

\newcommand{\mmid}{\mathbin{{\mid}{\mid}}}

\begin{document}

\title{Vector spaces, dimension, subspaces.} 
\author{Antoine Venant}
%\institute{UDS COLI}
\date{\today}
\maketitle


\begin{frame}{Examples from last time.}
  
  \begin{itemize}
  \item Define vectors as \emph{translations} in a euclidean geometric space ($\mathbb{R}^2$).
  \item One can add vectors (\emph{inner law}).
  \item One can multiply vectors by a scalar (\emph{outer law}).
  \item Addition and multipliation have some noticeable structure.
  \end{itemize}
\end{frame}

\begin{frame}
  \frametitle{Today:}
  The (more general) mathematical approach:
  \begin{itemize}
  \item First, define the \emph{space} itself.
  \item Def: a set with some kind of structure.
  \item Vectors? Just (abstract) elements of the set.
  \item What kind of structure?
  \item An (inner) addition, An (outer) product, and some assumption about these!
  \item Translations of the space are just a particular case.
  \end{itemize}  
\end{frame}

\begin{frame}
  \frametitle{Reminder: commutative group.}

  $\langle S, \bowtie \rangle$ is a commutative group iff all of the following hold (for any $x, y$):
  \begin{itemize}
  \item $S$ is some set and $\bowtie: S \times S \mapsto S$ (we'll use both infix and prefix notations: $x \bowtie y = \bowtie(x, y)$).
  \item $x \bowtie y = y \bowtie x$ (commmutativity).
  \item $x \bowtie (y \bowtie z) = (x \bowtie y) \bowtie z$ (associativity).
  \item $0 \in S$ s.t. $x \bowtie 0 = x$ (neutral element).
  \item There exists $-x$ s.t. $x + (-x) = 0$ (opposite element).
  \end{itemize}

  \begin{center}
    $y - x$ is syntactic sugar for $y + (-x)$.
  \end{center}
\end{frame}

\begin{frame}{Some properties (and exercises).}
  \begin{itemize}
  \item $0 \bowtie x = 0$.
  \item $x \bowtie z = x \rightarrow z = 0$. (Consequence: the neutral element is unique.)
  \item if $x \bowtie y = 0$ and $x \bowtie z = 0$ then $y = z$ (unique $-x$)
  \item $-0 = 0$
  \end{itemize}
\end{frame}


\begin{frame}{Vector space (over $\mathbb{R}$).}
  Formally:
  
  \begin{block}{Definition}
    $\langle V, {\mathbin{\textbf{+}}}, \circ \rangle$ is a \emph{vector space} (over $\mathbb{R}$) iff all of the following holds:
    \begin{itemize}
    \item ${\mathbin{\textbf{+}}}: V \times V \mapsto V$ and $\langle V, {\mathbin{\textbf{+}}} \rangle$ is a commutative group (we let ${\bf 0}$ denote its neutral element).
    \item ${\circ}: \mathbb{R} \times V \mapsto V$.
    \item $(\alpha \beta) \circ u = \alpha \circ (\beta \circ u))$. (Mixed associativity)
    \item $1 \circ u = u$. (Left neutrality)
    \item $(\alpha + \beta) \circ u = (\alpha \circ u) \mathbin{\textbf{+}} (\beta \circ u)$. (Right distributivity)
    \item $\alpha \circ (u\mathbin{\textbf{+}}v) = (\alpha \circ u) \mathbin{\textbf{+}} (\alpha \circ v)$. (Left distributivity)
    \end{itemize}
  \end{block}
\end{frame}

\begin{frame}
  \frametitle{Notational shortcuts:}
  \begin{itemize}
  \item We have two `addition' operators: ${+}$ on $\mathbb{R}$ and $\textbf{+}$ on vectors.
  \item We also have two `zeros' $0 \in \mathbb{R}$ and ${\bf 0}$ (null vector, neutral element of $\textbf{+}$).
  \item For simplicity, we will abuse notations: ${+}$ denotes either `addition' and $0$ denotes either `zero', depdending on context.
  \item Also, abreviate $\lambda \circ x$ with $\lambda x$ and let context disambiguate between $\circ$ and multiplication of reals.
  \item Give multiplication priority over $\circ$ and ${\circ}$ priority over ${+}$
  \end{itemize}

  \[
  \lambda\mu x + \alpha\beta y  = [(\lambda\mu)x] + [(\alpha\beta)y] = [(\lambda \times \mu) \circ x] {\mathbin{\textbf{+}}} [(\alpha \times \beta) \circ y.]
  \]
   
\end{frame}

\begin{frame}{More properties (and exercises).}
  \begin{itemize}
  \item $\lambda 0 = 0$
  \item $0x = 0$
  \item $\lambda x = 0$ iff $\lambda = 0$ or $x = 0$
  \item $-x = (-1)x$
  \end{itemize}
\end{frame}

\begin{frame}
  \frametitle{Examples}
  Some vector spaces...
  \begin{exampleblock}{$\mathbb{R}^n$}
    $\langle \mathbb{R}^n, +, \circ \rangle$ with component-wise operations:
    \begin{align*}
      &(x_1, \dots, x_n) + (y_1, \dots, y_n) = (x_1 + y_1, \dots, x_n + y_n)\\
      &\lambda \circ (x_1, \dots, x_n) = (\lambda x_1, \dots, \lambda x_n)\\
    \end{align*}
  \end{exampleblock}

  \begin{exampleblock}{Real-valued functions.}
    Given some set $X$ functions, $\langle \mathbb{R}^{X}, +, \cdot  \rangle$ with operations defined as:
    \begin{align*}
      &\forall x \in X,\,(f + g)(x) = f(x) + g(x)\\
      &\forall x \in X\, (\lambda \cdot f)(x) = \lambda f(x)\\
    \end{align*}
  \end{exampleblock}
\end{frame}

\begin{frame}{And more...}
  \begin{exampleblock}{Very non-exhaustive list.}
    \begin{itemize}
    \item Complex numbers.
    \item Polynomials with coefficient in $\mathbb{R}$
    \item formal series / sequences of real numbers.
    \item Continuous real-valued functions.
    \item \textbf{$(n,m)$-Matrices with real coefficients.} (later)
    \end{itemize}
  \end{exampleblock}
\end{frame}

\begin{frame}
  \frametitle{Just some notation.}
  In what follows and unless stated otherwise let $\langle V, +, \circ \rangle$ be an arbitrary vector space.
\end{frame}

\begin{frame}{Vector subspace.}
  \begin{block}{Definition}
    A \emph{vector subspace} of $V$ is a subset $W \subseteq V$ s.t. $\langle W, +, \circ \rangle$ is itself a vector space.
    In particular, this is equivalent to (for any $x,y \in W$, $\lambda \in \mathbb{R}$):
    \begin{itemize}
    \item $0 \in W$.
    \item $x + y \in W$.
    \item $\lambda x \in W$. 
    \end{itemize}
  \end{block}
\end{frame}

\begin{frame}{Intersection of subspaces}
  \begin{block}{Proposition}
    If $W_1$ and $W_2$ are two vector subspaces of $V$, then so is $W_1 \cap W_2$.
  \end{block}
\end{frame}

\begin{frame}{Sum of subspaces}
  \begin{alertblock}{\textbf{Not} a theorem}
    If $W_1$ and $W_2$ are vector subspaces of $V$, $W_1 \cup W_2$ is generally \textbf{not} a vector subspace of $V$. Counterexample?
  \end{alertblock}

  \begin{block}{Definition}
    If $W_1$ and $W_2$ are vector subspaces of $V$, then so is
    \[W_1 + W_2 = \{x + y \mid x \in W_1, y \in W_2\}\]
  \end{block}

  $W_1 + W_2$ is the smallest vector subspace that contains $W_1 \cup W_2$.
\end{frame}

\begin{frame}{Linear combinations.}
  \begin{block}{Definition}
    Let $\langle v_1, \dots, v_n \rangle$ be a sequence of vectors, $\lambda_1, \dots, \lambda_n \in \mathbb{R}$,
    \[\lambda_1 v_1 + \lambda_2 v_2 + \dots \lambda_n v_n\] is called a \emph{linear combination} of the $v_i$s.

   We let $\mathcal{L}(v_1, \dots, v_n) = \{ \lambda_1 v_1 + \lambda_2 v_2 + \dots \lambda_n v_n \mid \lambda_1, \dots, \lambda_n \in \mathbb{R}\}$ denote the set of linear combinations of $\langle v_1, \dots, v_n \rangle$.
  \end{block}

  \begin{block}{Proposition}
     $\mathcal{L}(v_1, \dots, v_n)$ is a vector subspace of $V$.
  \end{block}
    
\end{frame}

\begin{frame}{Linear independence.}
  \begin{block}{Definition}
    $\langle v_1, \dots, v_n \rangle$ is a sequence of \emph{linearly independent} vectors iff
    \[\lambda_1 v_1 + \dots + \lambda_n v_n = 0 \rightarrow \lambda_1 = \dots = \lambda_n = 0  \]
  \end{block}

  \begin{block}{Proposition}
    $\langle v_1, \dots, v_n \rangle$ is linear independent iff \[\forall i \in [|1, n|]\, v_i \notin \mathcal{L}(\langle v_1, \dots, v_{i-1}, v_{i+1}, \dots, v_n \rangle).\]
  \end{block}
\end{frame}

\begin{frame}{Generator.}
  \begin{block}{Definition}
    $\langle v_1, \dots, v_n \rangle$ is a \emph{generator} of $V$ iff $\mathcal{L}(\langle v_1, \dots, v_n \rangle) = V$.
  \end{block}
\end{frame}

\begin{frame}{Basis.}
  \begin{block}{Definition}
    A sequence of vectors $\langle v_1, \dots, v_n \rangle$ who is both linear independent and a generator is called a \emph{Basis} of $V$.
  \end{block}

  \begin{block}{Proposition}
    If $\langle v_1, \dots, v_n \rangle$ is a basis of $V$, then for any $v \in V$ there exists a {\bf unique} $\langle \lambda_1, \dots, \lambda_n \rangle \in \mathbb{R}^n$ such that
    \[ v = \lambda_1v_1 + \dots + \lambda_n v_n. \]
  \end{block}

\end{frame}

\begin{frame}{Example}
  \begin{exampleblock}{From last time:}
    \begin{itemize}
    \item $\langle (1,0), (0,1) \rangle$ is a basis of $\mathbb{R}^2$.
    \item $\langle (2,2), (-1,3) \rangle$ is a basis of $\mathbb{R}^2$.
    \end{itemize}
  \end{exampleblock}
\end{frame}

\begin{frame}{Canonical basis of $\mathcal{R}^n$}
    Let $\delta_{x,y}$ be one if $x = y$ and $0$ otherwise.\\
    Let $e^{(i)} = (e^{(i)}_1 \dots e^{(i)}_n)$ with $e^{(i)}_k = \delta(i, k)$. \footnote{$e^{(i)} = \langle \overbrace{0,\dots,0}^{(i-1) \textnormal{ times}}, 1, \overbrace{0,\dots,0}^{(n-i) \textnormal{ times}} \rangle$}
    \[\langle e^{(1)}, \dots, e^{(n)} \rangle \textnormal{ is a basis of } \mathbb{R}^n.\]
\end{frame}

\begin{frame}{Completion theorem.}
  \begin{block}{Theorem}
    If $\langle v_1, \dots, v_k \rangle$ is linearly independent and $\langle w_1, \dots, w_l \rangle$ is a sequence of vectors such that $\langle v_1, \dots, v_k, w_1, \dots, w_l \rangle$ is a generator of $V$, then there exists indices $i_1, \dots, i_m$ ($m \le l$) such that $\langle v_1, \dots, v_n, w_{i_1}, \dots, w_{i_m} \rangle$ is a basis of $V$.
  \end{block}
\end{frame}

\begin{frame}{Exchange lemma.}

  \begin{block}{Lemma}
  If $\langle v_1, \dots, v_n \rangle$ and $\langle w_1, \dots, w_m \rangle$ are two basis of $V$, then for any $i \in [|1, n|]$ there exist $j \in [|1, m|]$ such that
  \[\langle v_{1}, \dots, v_{i-1}, w_j, v_{i+1}, \dots v_n \rangle \textnormal{ is a basis of } V.\]
  \end{block}

  \begin{block}{Corollary}
    If $\langle v_1, \dots, v_n \rangle$ and $\langle w_1, \dots, w_m \rangle$ are two basis of $V$, then $n = m$.
  \end{block}
\end{frame}

\begin{frame}{Dimension}
  
  \begin{block}{Corollary}
    If $\langle v_1, \dots, v_n \rangle$ and $\langle w_1, \dots, w_m \rangle$ are two basis of $V$, then $n = m$.
  \end{block}

  \begin{block}{Definition}
    Let $V$ be a vector space such that $\langle v_1, \dots, v_n \rangle$ is a basis of $V$. The dimension of $V$ is defined as
    \[dim(V) = n.\]
    From the above corollary, this is a valid definition.
  \end{block}
\end{frame}


\begin{frame}{Dimension and subspaces}
  \begin{itemize}
  \item If $W$ is a vector subspace of $V$ and $V$ has a basis, then $dim(W) \le dim(V)$.
  \item If $W_1, W_2$ are vector subspaces of $V$ and $V$ has a basis, then $dim(W_1 + W_2) = dim(W_1) + dim(W_2) - dim(W_1 \cap W_2)$
  \end{itemize}
\end{frame}


\end{document}
