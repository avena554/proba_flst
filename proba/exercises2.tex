
\documentclass{beamer}

%\usepackage{listings}
%\usepackage[francais]{babel}
\usepackage[T1]{fontenc}
\usepackage[utf8]{inputenc}
%\usepackage{MyriadPro}
\usepackage{cabin}
\usepackage{graphicx}
\usepackage{array}
\usepackage{tikz}
\usetikzlibrary{positioning, backgrounds, shapes, chains, decorations.pathmorphing}

\usepackage{amsmath,amsthm,amssymb}  
\usepackage{stmaryrd}
%\usepackage{mdsymbol}
\usepackage{MnSymbol}
\usepackage{xcolor}
\usepackage{verbatim}
\usepackage{array}
%\usepackage{csquotes}



\usepackage[absolute,overlay]{textpos}
%\usepackage[texcoord,
%grid,gridcolor=red!10,subgridcolor=green!10,gridunit=pt]
%{eso-pic}



\useoutertheme{infolines}

\newcommand{\hidden}[1]{}

%colors
\definecolor{darkgreen}{rgb}{0,0.5,0}
\usebeamercolor{block title}
\definecolor{beamerblue}{named}{fg}
\usebeamercolor{alert block title}
\definecolor{beamealert}{named}{fg}

\renewcommand{\colon}{\!:\!}


\newcommand\paraitem{%
 \quad
 \makebox[\labelwidth][r]{%
 \makelabel{%
 \usebeamertemplate{itemize \beameritemnestingprefix item}}}\hskip\labelsep}

\newcommand{\mmid}{\mathbin{{\mid}{\mid}}}

\begin{document}

\title{Exercises} 
\author{Antoine Venant}
%\institute{UDS COLI}
\date{\today}
\maketitle


\begin{frame}{Exercise}

  Give the solution sets of the following systems of linear equations:

  \[
  \begin{array}{ccccccc}
     \phantom{1}x_1 & + & 2x_2 & + & 3x_3 & = & 1\\
    4x_1 & + & 5x_2 & + & 6x_3 & = & 2\\
    7x_1 & + & 8x_2 & + & 9x_3 & = & 3\\
    5x_1 & + & 7x_2 & + & 9x_3 & = & 4\\
  \end{array}
  \]


  \[
  \begin{array}{ccccccccc}
     \phantom{1}x_1 & - & \phantom{1}x_2 & + & 2x_3 & - &3x_4& = & \phantom{-}7\\
     4x_1 &  &  & + & 3x_3 & + &x_4& = & \phantom{-}9\\
     2x_1 & - & 5x_2 & + & x_3 &  & & = & -2\\
     3x_1 & - & \phantom{1}x_2 & - & x_3 & + &2x_4& = & -2\\
  \end{array}
  \]
  
\end{frame}


\begin{frame}{Exercise}
  Consider \[f: \begin{aligned} &\mathbb{R}^3 \mapsto \mathbb{R}^3\\&(x,y,z) \mapsto (x-y, x-z, z-y) \end{aligned}\]
    \begin{itemize}
    \item Show that $f$ is linear.
    \item Does there exists $(x,y,z)$ such that $f(x,y,z) = (1, 2, 3)$?
    \item Describe the set of $f^{-1}(2,1,1) = \{(x,y,z) \mid f(x,y,z) = (2,1,1)\}$.
    \item At what condition on $(a,b,c)$ do we have $f^{-1}(a,b,c) \neq \emptyset$. If this condition is granted, is $f^{-1}(a,b,c)$ finite?
    \end{itemize}
\end{frame}

\begin{frame}{Exercise}
  \begin{itemize}
  \item Show that a system of $n$ linear equations of $n$ variables
    \[
  \arraycolsep=1.4pt%\def\arraystretch{2.2}
  \begin{array}{ccccccccc}
    a^1_1 & x_1 & + & \dots & + & a^n_1 & x_n & = & b_1 \\
    a^1_2 & x_1 & + & \dots & + & a^n_2 & x_n & = & b_2\\
    \vdots&     &   &      &  &\vdots&     &   & \vdots\\
    a^1_n & x_1 & + & \dots & + &  a^n_n & x_n & = & b_n\\
  \end{array}
  \]
  admits a unique solution iff the associated homogeneous system
  \[
  \arraycolsep=1.4pt%\def\arraystretch{2.2}
  \begin{array}{ccccccccc}
    a^1_1 & x_1 & + & \dots & + & a^n_1 & x_n & = & 0 \\
    a^1_2 & x_1 & + & \dots & + & a^n_2 & x_n & = & 0\\
    \vdots&     &   &      &  &\vdots&     &   & \vdots\\
    a^1_n & x_1 & + & \dots & + &  a^n_n & x_n & = & 0\\
  \end{array}
  \]
  has a unique solution.
  \end{itemize}


\end{frame}


\begin{frame}{Exercise}
  \begin{itemize}
  \item Is it true that a system of $m < n$ equation of $n$ variables never has a unique solution?
  \item Is it true that a system of $m < n$ equation of $n$ variables always has at least one solution?
  \item Is it true that a system of $m > n$ equations of $n$ variables has at most one solution?
  \end{itemize} 
\end{frame}

\begin{frame}{Exercise}
  A box holding $1$cent, $5cent$ and $10$cent cointains contains thirteen coins with a total
  value of $83$ cents. How many coins of each type are in the box? 
\end{frame}
  


\end{document}
