
\documentclass{beamer}

%\usepackage{listings}
%\usepackage[francais]{babel}
\usepackage[T1]{fontenc}
\usepackage[utf8]{inputenc}
%\usepackage{MyriadPro}
\usepackage{cabin}
\usepackage{graphicx}
\usepackage{array}
\usepackage{tikz}
\usetikzlibrary{positioning, backgrounds, shapes, chains, decorations.pathmorphing}

\usepackage{amsmath,amsthm,amssymb}  
\usepackage{stmaryrd}
%\usepackage{mdsymbol}
\usepackage{MnSymbol}
\usepackage{xcolor}
\usepackage{verbatim}
\usepackage{array}
%\usepackage{csquotes}



\usepackage[absolute,overlay]{textpos}
%\usepackage[texcoord,
%grid,gridcolor=red!10,subgridcolor=green!10,gridunit=pt]
%{eso-pic}



\useoutertheme{infolines}

\newcommand{\hidden}[1]{}

%colors
\definecolor{darkgreen}{rgb}{0,0.5,0}
\usebeamercolor{block title}
\definecolor{beamerblue}{named}{fg}
\usebeamercolor{alert block title}
\definecolor{beamealert}{named}{fg}

\renewcommand{\colon}{\!:\!}


\newcommand\paraitem{%
 \quad
 \makebox[\labelwidth][r]{%
 \makelabel{%
 \usebeamertemplate{itemize \beameritemnestingprefix item}}}\hskip\labelsep}

\newcommand{\mmid}{\mathbin{{\mid}{\mid}}}

\begin{document}

\title{Linear Maps.} 
\author{Antoine Venant}
%\institute{UDS COLI}
\date{\today}
\maketitle


\begin{frame}{Previously:}
  \begin{itemize}
  \item Vector spaces
  \item Basis, Dimesion
  \item Subspaces, sum spaces, direct sums.
  \item Linear systems: checking linear independence, changing basis, intersecting hyperplans.
  \end{itemize}
\end{frame}

\begin{frame}{Today:}
  \begin{itemize}
  \item \emph{transformations} of a vector space.
  \item But not any transformation.
  \item Transformations that preserve the linear structure.
  \item Alternative take on linear systems.
  \end{itemize}
\end{frame}

\hidden{
\begin{exampleblock}{Example:}
  \begin{center}
      \input{automorphism}      
    \end{center}
\end{exampleblock}
}

\begin{frame}
  \frametitle{Example}
  \begin{exampleblock}{A linear map}
    \[f: \begin{aligned} \mathbb{R}^2 &\mapsto \mathbb{R}^2\\ (x,y) &\mapsto (x + y, y - x) \end{aligned}\]  
  \end{exampleblock}

  \begin{exampleblock}{Not a linear map}
    \[f: \begin{aligned} \mathbb{R}^2 &\mapsto \mathbb{R}^2\\ (x,y) &\mapsto (x \cdot cos(y), x \cdot sin(y)) \end{aligned}\]  
  \end{exampleblock}
\end{frame}


\begin{frame}{Two vector spaces.}
  In all the following we Let $V$ and $W$ be two vector spaces.
\end{frame}

\begin{frame}{Linear Map.}
  \begin{block}{Definition}
    A linear map (or homomorphism) from $V$ to $W$ is a function $f V \mapsto W$ such that for any $x, y \in V$ and $\lambda \in \mathbb{R}$
    \[\begin{aligned}
    f(x + y) = f(x) + f(y) \textnormal{and}\\
    f(\lambda x) = \lambda f(x).
    \end{aligned}\]
    $\mathcal{H}(V, W)$ is the set of all linear maps from $V$ to $W$.
  \end{block}
\end{frame}

\begin{frame}{Properties and remarks}
  \begin{block}{Remark}
    \begin{itemize}
    \item The \emph{identity function} $id_{V}: x \mapsto x$ is a linear map from $V$ to itself.
    \item The composition $f \circ g$ of two linear maps ($[f \circ g](x) = f(g(x))$) is also linear map.
    \end{itemize}
  \end{block}

  \begin{block}{Proposition}
    With the usual definition for function addition and scalar multiplication
    \[\begin{aligned}
    &[f + g] : x \mapsto f(x) + g(x)\\
    &[\lambda f] : x \mapsto \lambda f(x),
    \end{aligned}\]
    we have that $\langle \mathcal{H}, +, \dot  \rangle$ is a vector space.
  \end{block}
\end{frame}

\begin{frame}
  \frametitle{Proposition}
  Let $\langle v_1 \dots v_n \rangle$ be a basis of $V$ and $\langle w_1, \dots, w_n \rangle$ be a sequence of $n$ vectors of $W$ (not necessarily a basis). There exists a unique linear map $f: V \mapsto W$ such that
  \[f(v_1) = w_1 \textnormal{ and } f(v_2) = w_2 \textnormal{ and } \dots \textnormal{ and } f(v_n) = w_n\]
\end{frame}




\begin{frame}{Kernel and Image}
  Let $f: V \mapsto W$ be a linear map.

  \begin{block}{Definition}
    The image of $f$ is the set of images of all vectors of $V$:
    \[Im(f) = \{ f(x) \mid x \in V \} \]
  \end{block}

  \begin{block}{Definition}
    The kernel of $f$ is the set of vectors that are mapped to $0$\footnote{$0$ denotes indifferently the null vector of $W$ and the one of $V$. Which one is reffered to should be clear from context -- in this case the $0$ of $W$}.
    \[Ker(f) = \{ x \mid f(x) = 0\}\]
  \end{block}
\end{frame}

\begin{frame}
\begin{exampleblock}{Example}
  \[f: \begin{aligned} \mathbb{R}^2 &\mapsto \mathbb{R}^2\\ (x,y) &\mapsto (x + y, y - x) \end{aligned}\]

  \begin{center}
    \input{kernim}
  \end{center}
  
\end{exampleblock}

\end{frame}

\begin{frame}
  \frametitle{Properties}
  Let $f: V \mapsto W$ be a linear map. 
  \begin{block}{Proposition}
    \begin{itemize}
    \item $Im(f)$ is a vector subspace of $W$.
    \item $Ker(f)$ is a vector subspace of $V$. 
    \end{itemize}
  \end{block}

  \begin{block}{Proposition}
    $f$ is injective iff $Ker(f) = \{0\}$ (injective means no two points have the same image, $f(x) = f(y) \rightarrow x = y$)
  \end{block}
\end{frame}


\begin{frame}
  \frametitle{Vocabulary}
  \begin{itemize}
  \item When $V = W$, $f$ is called an \emph{endomorphism}.
  \item A bijective linear map is called an \emph{isomorphism}.
  \item A bijective endomorphism is an \emph{automorphism}.
  \item Two vector spaces are \emph{isomorphic} if there exists an isomorphism between them.
  \end{itemize}
\end{frame}


\begin{frame}
  \frametitle{Isomorphisms}
  \begin{block}{Proposition}
    \begin{itemize}
    \item The converse of an isomorphism is also an isomorphism.
    \item The composition of two isomorphisms is also a linear map.
    \end{itemize}
  \end{block}

  \begin{block}{Proposition}
    Let $f: V \mapsto W$ be a linear map. If $\langle v_1, \dots, v_n \rangle$ is a basis of $V$ then $f$ is an isomorphism iff $\langle f(v_1), \dots, f(v_n) \rangle$ is a basis of $W$.
  \end{block}

  \begin{block}{Corollary}
    Any two $n-$dimenstional vector spaces are isomorphic.
  \end{block}
\end{frame}


\begin{frame}{Reminder}
  \begin{block}{Completion theorem}
    
    If $\langle v_1, \dots, v_k \rangle$ is linearly independent and $\langle w_1, \dots, w_l \rangle$ is a sequence of vectors such that $\langle v_1, \dots, v_k, w_1, \dots, w_l \rangle$ is a generator of $V$, then there exists indices $i_1, \dots, i_m$ ($m \le l$) such that $\langle v_1, \dots, v_n, w_{i_1}, \dots, w_{i_m} \rangle$ is a basis of $V$.
  \end{block}
  
\end{frame}

\begin{frame}
  \frametitle{Linear maps and dimensions}
  Let $V$ and $W$ be {\bf finite dimensional} vector spaces and $f: V \mapsto W$ be a linear map.

  \begin{block}{Theorem}
    \[dim(Ker(f)) + dim(Im(f)) = dim(V)\]
  \end{block}

  \begin{block}{Corollary}
    A linear map is surjective iff it is injective.
  \end{block}
\end{frame}




\end{document}
