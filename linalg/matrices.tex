
\documentclass{beamer}

%\usepackage{listings}
%\usepackage[francais]{babel}
\usepackage[T1]{fontenc}
\usepackage[utf8]{inputenc}
%\usepackage{MyriadPro}
\usepackage{cabin}
\usepackage{graphicx}
\usepackage{array}
\usepackage{tikz}
\usetikzlibrary{positioning, backgrounds, shapes, chains, decorations.pathmorphing}

\usepackage{amsmath,amsthm,amssymb}  
\usepackage{stmaryrd}
%\usepackage{mdsymbol}
\usepackage{MnSymbol}
\usepackage{xcolor}
\usepackage{verbatim}
\usepackage{array}
\usepackage{multirow}
%\usepackage{csquotes}



\usepackage[absolute,overlay]{textpos}
%\usepackage[texcoord,
%grid,gridcolor=red!10,subgridcolor=green!10,gridunit=pt]
%{eso-pic}



\useoutertheme{infolines}

\newcommand{\hidden}[1]{}

%colors
\definecolor{darkgreen}{rgb}{0,0.5,0}
\usebeamercolor{block title}
\definecolor{beamerblue}{named}{fg}
\usebeamercolor{alert block title}
\definecolor{beamealert}{named}{fg}

\renewcommand{\colon}{\!:\!}


\newcommand\paraitem{%
 \quad
 \makebox[\labelwidth][r]{%
 \makelabel{%
 \usebeamertemplate{itemize \beameritemnestingprefix item}}}\hskip\labelsep}

\newcommand{\mmid}{\mathbin{{\mid}{\mid}}}

\begin{document}

\title{Matrices.} 
\author{Antoine Venant}
%\institute{UDS COLI}
\date{\today}
\maketitle


\begin{frame}{Previously:}
  \begin{itemize}
  \item Linear map between vector spaces $V$ and $W$.
  \item \emph{Determined by the image of the vectors of a basis of $V$}.
  \end{itemize}
\end{frame}

\begin{frame}{Today:}
  \begin{itemize}
  \item Use this property to define a representation suitable for calculus
  \item Matrices.
  \item Matrices are representation of linear maps w.r.t. a given basis $V$ and a given basis of $W$.
  \end{itemize}
\end{frame}

\hidden{
\begin{exampleblock}{Example:}
  \begin{center}
      \begin{tikzpicture}[scale = 0.7]
  \draw[help lines, color=gray!30, dashed] (-2.9,-2.9) grid (2.9,2.9);
\draw[->,ultra thick] (-3,0)--(3,0) node[right]{$x$};
\draw[->,ultra thick] (0,-3)--(0,3) node[above]{$y$};

\foreach \x in  {-2,-1,0,1,2}
\draw[shift={(\x,0)},color=black] (0pt,3pt) -- (0pt,-3pt);
\foreach \x in {-2,-1,0,1,2}
\draw[shift={(\x,0)},color=black] (0pt,0pt) -- (0pt,-3pt) node[below] 
     {$\x$};


\foreach \y in  {-2,-1,1,2}
\draw[shift={(0, \y)},color=black] (3pt,0pt) -- (-3pt,-0pt);
\foreach \y in {-2,-1,1,2}
\draw[shift={(0, \y)},color=black] (0pt,0pt) -- (-3pt,0pt) node[right] 
     {$\y$};
     
     
     %\draw[blue, ->, ultra thick, dashed] (0,0) -- (2, 2);
     %\draw[blue, ->, ultra thick, dashed] (0,0) -- (-1, 3);
     \draw[red, ->, ultra thick] (0,0) -- (0,1);
     \draw[red, ->, ultra thick] (0,0) -- (1,0);
     %\draw[red, ->, thick, dashed] (1,1) -- (0.66666,2);
     %\draw[red, ->, thick, dashed] (-0.33333,1) -- (0.66666,2);
     %\draw[darkgreen, ->, ultra thick] (0,0) -- (0.66666,2) node[above right]{$\frac{1}{2}\overline{(2,2)} + \frac{1}{3}\overline{(-1,3)}$};

     
     \begin{scope}[xshift = 10cm]

       \draw[help lines, color=gray!30, dashed] (-2.9,-2.9) grid (2.9,2.9);
\draw[->,ultra thick] (-3,0)--(3,0) node[right]{$x$};
\draw[->,ultra thick] (0,-3)--(0,3) node[above]{$y$};

\foreach \x in  {-2,-1,0,1,2}
\draw[shift={(\x,0)},color=black] (0pt,3pt) -- (0pt,-3pt);
\foreach \x in {-2,-1,0,1,2}
\draw[shift={(\x,0)},color=black] (0pt,0pt) -- (0pt,-3pt) node[below] 
     {$\x$};


\foreach \y in  {-2,-1,1,2}
\draw[shift={(0, \y)},color=black] (3pt,0pt) -- (-3pt,-0pt);
\foreach \y in {-2,-1,1,2}
\draw[shift={(0, \y)},color=black] (0pt,0pt) -- (-3pt,0pt) node[right] 
     {$\y$};
     
     
     %\draw[blue, ->, ultra thick, dashed] (0,0) -- (2, 2);
     %\draw[blue, ->, ultra thick, dashed] (0,0) -- (-1, 3);
     \draw[red, ->, ultra thick] (0,0) -- (1,1);
     \draw[red, ->, ultra thick] (0,0) -- (-0.3333,1);
     %\draw[red, ->, thick, dashed] (1,1) -- (0.66666,2);
     %\draw[red, ->, thick, dashed] (-0.33333,1) -- (0.66666,2);
     %\draw[darkgreen, ->, ultra thick] (0,0) -- (0.66666,2) node[above right]{$\frac{1}{2}\overline{(2,2)} + \frac{1}{3}\overline{(-1,3)}$};


     \end{scope}
     
\end{tikzpicture}
      
    \end{center}
\end{exampleblock}
}

\begin{frame}
  \frametitle{Example}
  \begin{exampleblock}{A linear map}
    \[f: \begin{aligned} \mathbb{R}^2 &\mapsto \mathbb{R}^2\\ (x,y) &\mapsto (x - y, y - x) \end{aligned}\]

    Two basis of $\mathbb{R}^2 :$
    \[\begin{aligned}
    &\mathcal{E} = \langle e_1, e_2 \rangle = \langle (1,0), (0,1) \rangle \\ 
    &\mathcal{B} = \langle \alpha, \beta \rangle = \langle (1,1), (-1, 1) \rangle\\
    \end{aligned}
    \]
    \[
    \begin{aligned}
      &f(e_1) = (1,-1) = {\color{blue} \langle 1, -1 \rangle_{\mathcal{E}}} = {\color{darkgreen} \langle 0, -1\rangle_{\mathcal{B}}}\\
      &f(e_2) = (-1, 1) = {\color{blue} \langle -1, 1 \rangle_{\mathcal{E}}} = {\color{darkgreen} \langle 0, 1 \rangle_{\mathcal{B}}} \\      
    \end{aligned}
    \]


    \[f:
    \begin{array}{|ccc|}
      f({\color{red}e_1}) & f({\color{red}e_2}) & \\
      \color{blue}\phantom{-}1 & \color{blue} -1 & \color{blue} e_1\\
      \color{blue}-1 & \color{blue} \phantom{-}1 & \color{blue} e_2
    \end{array}_{{\color{red}\mathcal{E}}, {\color{blue}\mathcal{E}}}
    \equiv
    \begin{array}{|ccc|}
      f({\color{red}e_1}) & f({\color{red}e_2}) & \\
      \color{darkgreen}\phantom{-}0 & \color{darkgreen} \phantom{-}0 & \color{darkgreen} \alpha\\
      \color{darkgreen} -1 & \color{darkgreen}\phantom{-}1 & \color{darkgreen} \beta
    \end{array}_{{\color{red}\mathcal{E}}, {\color{darkgreen} \mathcal{B} }}
    \]

    
  \end{exampleblock}
\end{frame}


\begin{frame}
  {Matrix}
  Let $m, n \in \mathbb{N}$.
  \begin{block}{Definition}
    An $(n,m)$ matrix is a sequence of $n \times m$ real coefficients $a_{i,j}$ with $1 \le i \le n, 1 \le j \le m$. We can organize them visually in the following way:

    \[\begin{array}{|ccc|} a_{1,1} & \dots & a_{1,m}\\ \vdots & & \vdots\\ a_{n,1} & \dots & a_{n, m} \end{array}\]

    $\mathcal{M}(n,m)$: set of all $n,m$ matrices.
  \end{block}
\end{frame}


\begin{frame}{Matrix $\Rightarrow$ linear map}
  \begin{block}{Definition}
    To any $n,m$ matrix $A = \langle a_{i,j} \rangle$ we associate an application $\hat A : \left \{ \begin{array}{l} \mathbb{R}^m \mapsto \mathbb{R}^n \\ X \rightarrow A \times X \end{array} \right .$ as follows:
    \[
    \begin{array}{|ccc|} a_{1,1} & \dots & a_{1,m}\\ \vdots & & \vdots\\ a_{n,1} & \dots & a_{n, m} \end{array} \times \underbrace{\left ( \begin{array}{c} x_1 \\ \vdots \\ x_m \end{array} \right )}_{X} = \left ( \begin{array}{c} a_{1,1}x_1 + \dots + a_{1,m}x_m \\ \vdots \\ a_{n,1}x_1 + \dots + a_{n,m} x_m \end{array} \right )  
    \]  
  \end{block}

  \begin{block}{Proposition}
    The associated application $\hat A$ defined above is a linear map.
  \end{block}
\end{frame}

\begin{frame}{Example}
  \begin{exampleblock}{$2,2$ matrix}
    \[ \begin{array}{|cc|}
      \phantom{-}1 &  -1 \\
      -1 & \phantom{-}1 
    \end{array} \times \left ( \begin{array}{c} x \\ y \end{array} \right ) = \left ( \begin{array}{c} x - y \\ y - x \end{array} \right )  
    \]
  \end{exampleblock}
\end{frame}

\begin{frame}{Reminder: canonical basis}
  \begin{block}{Canonical basis of $\mathcal{R}^m$}
    Letting $e_i = \langle \overbrace{0,\dots,0}^{(i-1) \textnormal{ times}}, 1, \overbrace{0,\dots,0}^{(m-i) \textnormal{ times}} \rangle$ we have: \[\langle e_1, \dots, e_m \rangle \textnormal{ is the canonical basis of } \mathbb{R}^m.\]
  \end{block}
\end{frame}

\begin{frame}
  \frametitle{Linear map $\Rightarrow$ matrix}
  \begin{block}{Converse}
    Let $f : \mathbb{R}^m \rightarrow \mathbb{R}^n$ linear. There exists a unique matrix $A \in \mathcal{M}(n,m)$ such that $\hat A = f$.
  \end{block}

  \begin{block}{proof}
    \begin{itemize}
    \item We know that $f$ is uniquely determined by the images of the canonical basis of $\mathbb{R}^m$: $\langle e_1, \dots, e_m \rangle$.
    \item by def, $f(e_1), \dots, f(e_m) \in \mathbb{R}^n$.
    \item So let $\langle a_{1,1}, \dots, a_{n,1} \rangle = f(e_1) \dots \langle a_{1, m} \dots a_{m, n} \rangle = f(e_m)$.
    \item Then $A = \begin{array}{|ccc|} a_{1,1} & \dots & a_{1,m}\\ \vdots & & \vdots\\ a_{n,1} & \dots & a_{n, m} \end{array}$ is such that $\hat A = f$:
    \item suffices to check $\hat A(e_1) = A \times e_1 = f(e_1) \dots \hat A(e_m) = A \times e_m = f(e_m)$.
    \end{itemize}
  \end{block}
  
\end{frame}


\begin{frame}
  \frametitle{Operations on matrices}

  \begin{block}{Scalar multiplication.}
    For $\lambda \in \mathbb{R}$:
    \[
    \lambda \cdot
    \begin{array}{|ccc|}
    a_{1,1} & \dots & a_{1,m}\\ \vdots & & \vdots\\ a_{n,1} & \dots & a_{n, m}
    \end{array} =
    \begin{array}{|ccc|}
      \lambda a_{1,1} & \dots & \lambda a_{1,m}\\ \vdots & & \vdots\\ \lambda a_{n,1} & \dots & \lambda a_{n, m}
    \end{array}
    \]
  \end{block}

  \begin{block}{Addition}
    Defined for two $n,m$ matrices: \alert{\textbf{addition of two matrices of different sizes is undefined}}
    \[
    \begin{array}{|ccc|}
      a_{1,1} & \dots & a_{1,m}\\ \vdots & & \vdots\\ a_{n,1} & \dots & a_{n, m}
    \end{array}
    +
    \begin{array}{|ccc|}
      b_{1,1} & \dots & b_{1,m}\\ \vdots & & \vdots\\ b_{n,1} & \dots & b_{n, m}
    \end{array}
    =
    \begin{array}{|ccc|}
      a_{1,1} + b_{1,1} & \dots & a_{1,m} + b_{1,m}\\ \vdots & & \vdots\\ a_{n,1} + b_{n,1} & \dots & a_{n, m} + b_{n,m}
    \end{array}
    \]
  \end{block}
  
\end{frame}

\begin{frame}{Compatibility with ops on linear maps.}
  \begin{block}{Proposition}
    \[
    \begin{aligned}
      &\widehat{A + B} = \hat A + \hat B &\widehat{\lambda A} = \lambda \cdot \hat A
    \end{aligned}
    \]
  \end{block}

  \begin{exampleblock}{Example}
    \[\begin{aligned}
    §\left ( \lambda \cdot
    \begin{array}{|cc|}
      \phantom{-}1 &  -1 \\
      -1 & \phantom{-}1 
    \end{array} \right ) \times \left ( \begin{array}{c} x \\ y \end{array} \right ) = \begin{array}{|cc|}
      \phantom{-}\lambda &  -\lambda \\
      -\lambda & \phantom{-}\lambda
    \end{array} \times \left ( \begin{array}{c} x \\ y \end{array} \right ) &= \left ( \begin{array}{c} \lambda(x - y) \\ \lambda(y - x) \end{array} \right ) \\&=
    \lambda \cdot \left (
    \begin{array}{|cc|}
      \phantom{-}1 &  -1 \\
      -1 & \phantom{-}1 
    \end{array}
    \times \left ( \begin{array}{c} x \\ y \end{array} \right )  \right )
    \end{aligned}
    \]
  \end{exampleblock}
\end{frame}


\begin{frame}{Properties}

  \begin{block}{Vector space.}
    $\langle \mathcal{M}(n, m), \cdot, {+}\rangle$ is a vector space. In particular:
    \begin{itemize}
    \item $0$: $\begin{array}{|ccc|} 0 & \dots & 0\\ \vdots & & \vdots\\ 0 & \dots & 0 \end{array}$.
    \item $\lambda (A + B) = \lambda A + \lambda B$ (left distrib,)
    \item $(\lambda + \mu) A = \lambda A + \mu A$ (rigth distrib.)
    \item $(\lambda \mu) A$ = $\lambda (\mu A)$ (mixed assoc.)
    \end{itemize}
  \end{block}

  \begin{block}{Matrices $\leftrightarrow$ Linear Maps.}
    Let $\mahtcal{H}(\mathbb{R}^n, \mathbb{R}^m)$ denote the set of all linear maps between $\mathbb{R}^n$ and $\mathbb{R}^m$.\\
    $A \mapsto \hat{A}$ is an isomorphism between $\mathcal{M}(n,m)$ and $\textsf{H}(\mathbb{R}^n, \mathbb{R}^m)$.
  \end{block}

\end{frame}

\begin{frame}{Product of matrices}

  \begin{block}{Definition}
    \alert{$A \times B$ defined only when $A$ $n,m$ matrix and $B$ $m,l$ matrix.} then: \[{\color{red}C} = {\color{darkgreen} A} \times {\color{blue} B}\textnormal{  is an $n,l$ matrix} \].
    
    \[
    \begin{array}{cccccc}
      & \multirow{2}{*}{$\times$} & & \color{blue}b_{1,1} &\color{blue} \dots & \color{blue} b_{1,l}\\
      &                           & & \vdots &       & \vdots\\
      &                           & &\color{blue} b_{m,1}  & \dots & \color{blue}b_{m, l}\\
      \color{darkgreen}a_{1,1} &\color{darkgreen} \dots & \color{darkgreen}a_{1,m} & \phantom{\color{red}a_{1,1}b_{1,1} + \dots a_{1,m}b_{m,1}}  & \phantom{\color{red}\dots}  & \phantom{\color{red}a_{1,1}b_{1, l} + \dots + a_{1, m}b_{m, l}}\\
      \color{darkgreen}\vdots &       & \color{darkgreen}\vdots  &\phantom{\color{red}\vdots}  &  &\phantom{\color{red}\vdots} \\
      \color{darkgreen}a_{n,1} & \color{darkgreen}\dots & \color{darkgreen}a_{n, m}& \phantom{\color{red}a_{n,1}b_{1,1} + \dots a_{n,m}b_{m,1}}  & \phantom{\color{red}\dots}  & \phantom{\color{red}a_{n,1}b_{1, l} + \dots + a_{n, m}b_{m, l}}\\
    \end{array}
    \]
    
  \end{block}
  
\end{frame}

\begin{frame}{Product of matrices}

  \begin{block}{Definition}
    \alert{$A \times B$ defined only when $A$ $n,m$ matrix and $B$ $m,l$ matrix.} then: \[{\color{red}C} = {\color{darkgreen} A} \times {\color{blue} B}\textnormal{  is an $n,l$ matrix} \].
    
    \[
    \begin{array}{cccccc}
      & \multirow{2}{*}{$\times$} & & \bf \color{blue}b_{1,1} &\color{blue} \dots & \color{blue} b_{1,l}\\
      &                           & & \bf \vdots &       & \vdots\\
      &                           & & \bf \color{blue} b_{m,1}  & \dots & \color{blue}b_{m, l}\\
      \bf \color{darkgreen}a_{1,1} &\bf \color{darkgreen} \dots & \bf \color{darkgreen}a_{1,m} & \color{red}a_{1,1}b_{1,1} + \dots a_{1,m}b_{m,1}  & \phantom{\color{red}\dots}  & \phantom{\color{red}a_{1,1}b_{1, l} + \dots + a_{1, m}b_{m, l}}\\
      \color{darkgreen}\vdots &       & \color{darkgreen}\vdots  &\phantom{\color{red}\vdots}  &  &\phantom{\color{red}\vdots} \\
      \color{darkgreen}a_{n,1} & \color{darkgreen}\dots & \color{darkgreen}a_{n, m}& \phantom{\color{red}a_{n,1}b_{1,1} + \dots a_{n,m}b_{m,1}}  & \phantom{\color{red}\dots}  & \phantom{\color{red}a_{n,1}b_{1, l} + \dots + a_{n, m}b_{m, l}}\\
    \end{array}
    \]
    
  \end{block}
  
\end{frame}





\begin{frame}{Product of matrices}

  \begin{block}{Definition}
    \alert{$A \times B$ defined only when $A$ $n,m$ matrix and $B$ $m,l$ matrix.} then: \[{\color{red}C} = {\color{darkgreen} A} \times {\color{blue} B}\textnormal{  is an $n,l$ matrix} \].
    
    \[
    \begin{array}{cccccc}
      & \multirow{2}{*}{$\times$} & & \bf \color{blue}b_{1,1} &\color{blue} \dots & \color{blue} b_{1,l}\\
      &                           & & \bf \vdots &       & \vdots\\
      &                           & &\bf \color{blue} b_{m,1}  & \dots & \color{blue}b_{m, l}\\
      \color{darkgreen}a_{1,1} &\color{darkgreen} \dots & \color{darkgreen}a_{1,m} & \color{red}a_{1,1}b_{1,1} + \dots a_{1,m}b_{m,1}  & \phantom{\color{red}\dots}  & \phantom{\color{red}a_{1,1}b_{1, l} + \dots + a_{1, m}b_{m, l}}\\
      \color{darkgreen}\vdots &       & \color{darkgreen}\vdots  &\color{red}\vdots  &  &\phantom{\color{red}\vdots} \\
      \bf \color{darkgreen}a_{n,1} & \bf \color{darkgreen}\dots & \bf \color{darkgreen}a_{n, m}& \color{red}a_{n,1}b_{1,1} + \dots a_{n,m}b_{m,1}  & \phantom{\color{red}\dots}  & \phantom{\color{red}a_{n,1}b_{1, l} + \dots + a_{n, m}b_{m, l}}\\
    \end{array}
    \]
    
  \end{block}
  
\end{frame}




\begin{frame}{Product of matrices}

  \begin{block}{Definition}
    \alert{$A \times B$ defined only when $A$ $n,m$ matrix and $B$ $m,l$ matrix.} then: \[{\color{red}C} = {\color{darkgreen} A} \times {\color{blue} B}\textnormal{  is an $n,l$ matrix} \].
    
    \[
    \begin{array}{cccccc}
      & \multirow{2}{*}{$\times$} & &  \color{blue}b_{1,1} &\color{blue} \dots & \bf\color{blue} b_{1,l}\\
      &                           & &  \vdots &       & \bf \vdots\\
      &                           & & \color{blue} b_{m,1}  & \dots & \bf \color{blue}b_{m, l}\\
      \bf \color{darkgreen}a_{1,1} &\bf \color{darkgreen} \dots & \bf \color{darkgreen}a_{1,m} & \color{red}a_{1,1}b_{1,1} + \dots a_{1,m}b_{m,1}  & \color{red}\dots  & \color{red}a_{1,1}b_{1, l} + \dots + a_{1, m}b_{m, l}\\
      \color{darkgreen}\vdots &       & \color{darkgreen}\vdots  &\color{red}\vdots  &  &\phantom{\color{red}\vdots} \\
      \color{darkgreen}a_{n,1} &  \color{darkgreen}\dots &  \color{darkgreen}a_{n, m}& \color{red}a_{n,1}b_{1,1} + \dots a_{n,m}b_{m,1}  & \phantom{\color{red}\dots}  & \phantom{\color{red}a_{n,1}b_{1, l} + \dots + a_{n, m}b_{m, l}}\\
    \end{array}
    \]
    
  \end{block}
  
\end{frame}


\begin{frame}{Product of matrices}

  \begin{block}{Definition}
    \alert{$A \times B$ defined only when $A$ $n,m$ matrix and $B$ $m,l$ matrix.} then: \[{\color{red}C} = {\color{darkgreen} A} \times {\color{blue} B}\textnormal{  is an $n,l$ matrix} \].
    
    \[
    \begin{array}{cccccc}
      & \multirow{2}{*}{$\times$} & &  \color{blue}b_{1,1} &\color{blue} \dots & \bf\color{blue} b_{1,l}\\
      &                           & &  \vdots &       & \bf \vdots\\
      &                           & & \color{blue} b_{m,1}  & \dots & \bf \color{blue}b_{m, l}\\
       \color{darkgreen}a_{1,1} & \color{darkgreen} \dots &  \color{darkgreen}a_{1,m} & \color{red}a_{1,1}b_{1,1} + \dots a_{1,m}b_{m,1}  & \color{red}\dots  & \color{red}a_{1,1}b_{1, l} + \dots + a_{1, m}b_{m, l}\\
      \color{darkgreen}\vdots &       & \color{darkgreen}\vdots  &\color{red}\vdots  &  &\phantom{\color{red}\vdots} \\
      \bf \color{darkgreen}a_{n,1} &  \bf \color{darkgreen}\dots &  \bf \color{darkgreen}a_{n, m}& \color{red}a_{n,1}b_{1,1} + \dots a_{n,m}b_{m,1}  & \color{red}\dots  & \color{red}a_{n,1}b_{1, l} + \dots + a_{n, m}b_{m, l}\\
    \end{array}
    \]
    
  \end{block}
  
\end{frame}

\begin{frame}{Formal definition}
  \begin{block}{Definition}
    For $A = (a_{i,j}) \in \mathcal{M}(n,m)$, $B = (b_{i,j}) \in \mathcal{M}(m, l)$,
    \[A \times B = (c_{i,j}) \in \mathcal{M}(n, l) \] where
    \[\forall i \in [1, n], j \in [1,l]\, c_{i,j} = \sum^{m}_{k = 1} a_{i, k}b_{k, j}\]
  \end{block}
\end{frame}

\begin{frame}{Example}
  \[ \underbrace{\begin{array}{|cc|} 1 & 0 \\ 0 & 1 \\ 0 &  0  \end{array}}_{(\textcolor{darkgreen}{3},2)} \times \underbrace{\begin{array}{|cc|} \phantom{-}1 & -1 \\ -1 & \phantom{-}1 \end{array}}_{2,\textcolor{blue}{2}} = \underbrace{\begin{array}{|cc|} \phantom{-}1 & -1 \\ -1 & \phantom{-}1 \\ \phantom{-}0 & \phantom{-}0  \end{array}}_{(\textcolor{darkgreen}{3}, \textcolor{blue}{2} )}    \]
\end{frame}

\begin{frame}{Properties}
  \begin{itemize}
  \item $(A \times B) \times C = A \times (B \times C)$ [Associativity]
  \item $(A + B) \times C = (A \times C) + (B \times C)$ [right distribuvity]
  \item $A \times (B + C) = (A \times B) + (A \times C)$ [left distributivity]
  \item $A \times X$ coincide with previous def. when $X$ is seen as a $m, 1$ matrix, $A$ is a $n,m$ matrix.
  \end{itemize}

  \begin{alertblock}{WARNING!}
    Matrix product is {\bf not} commutative in general:
    \[ A \times B \neq B \times A \textnormal{ in general.}\] 
  \end{alertblock}
\end{frame}

\begin{frame}{Identity}
  \begin{block}{Identity matrix}
    The ($n,n$) matrix
    \[ \begin{array}{|ccccc|}
      1 & 0 & \dots & 0 &0\\
      0 & 1 &   \dots &0 & 0\\
      \vdots  & \vdots & \ddots & \vdots & \vdots\\
      0 & 0  & \dots & 1 & 0\\
      0 & 0  & \dots & 0 & 1\\
      \end{array}
      \]  is called the identity $(n,n)$ matrix, $Id_{(n,n)}$.  
  \end{block}
  

  \begin{itemize}
  \item For any $(n,n)$ matrix $A$, $Id_{n} \times A = A \times Id_{n} = A$.
  \item Also ,for any $(n,l)$ matrix $A$, $Id{n} \times A = A$.
  \item In particular, for a $(n,1)$ column vector $X$, $Id_{n} \times X = X$.
  \item So $\widehat{Id_{n}}: \begin{array}{l} \mathbb{R}^n \mapsto \mathbb{R}^n \\ X \mapsto X \end{array}$
  \end{itemize}
  
\end{frame}

\begin{frame}
  \frametitle{further remarks}
  \begin{block}{Ring of square matrices}
    In algebra, the structure of $\langle \mathcal{M}(n,n), {+}, {\times} \rangle$ is called a \emph{ring} (another ring is for instance $\langle \mathbb{N}, {+}, {\times} \rangle$).
  \end{block}
  
  \begin{block}{Non integral domain}
    The ring of square matrices does \textbf{not} have an \emph{integral domain} which means there are non-zero matrices that multiply to zero:
    \[
    \begin{pmatrix}
      0 & 1\\
      0 & 1
    \end{pmatrix}
    \times
    \begin{pmatrix}
      1 & 1\\
      0 & 0
    \end{pmatrix}
    =
    \begin{pmatrix}
      0 & 0\\
      0 & 0
    \end{pmatrix}
    \]
  \end{block}
\end{frame}


\begin{frame}{Matrix product and composition.}
  \begin{block}{Theorem}
    Let $g: \mathbb{R}^l \mapsto \mathbb{R}^m$ be a linear map and $f \mathbb{R}^m \mapsto \mathbb{R}^n$ be a linear map. Let $A \in \mathcal{M}(n,m), B \in \mathcal{M}(m, l)$ such that $\hat A = f$ and $\hat B = g$. Then
    \[ \widehat{A \times B} = f \circ g \] or equivalently, for any $X \in \mathbb{R}^l$
    \[A \times B \times X = f(g(X))\]
  \end{block}
\end{frame}


\begin{frame}{Representing maps between arbitrary vector spaces.}
  \begin{itemize}
  \item So far: $n,m$ matrix $\leftrightarrow$ linear map between spaces $\mathbb{R}^m$ and $\mathbb{R}^n$.
  \item What about mapping between another $m$ dimensional space and another $n$ dimensional space?
  \item Possible, {\bf but only relative to two basis!}
  \end{itemize}
\end{frame}

\begin{frame}{Representation of a linear map.}
  Let $f: V \mapsto W$ be a linear map. Let $\mathcal{B} = \langle v_1, \dots, v_m \rangle$ be a basis of $V$ and $\mathcal{B'} = \langle w_1, \dots, w_n \rangle$ be a basis of $W$. Decompose each $f(v_i)$ w.r.t. $\mathcal{B}'$:

  \[\begin{aligned}
  f(v_1) &= a_{1,1} w_1 + \dots + a_{n,1} w_n = \langle a_{1,1}, \dots, a_{n,1} \rangle_{\mathcal{B}'}\\
  &\vdots\\
  f(v_m) & = a_{1,m} w_1 + \dots + a_{n,m} w_n = \langle a_{1,m}, \dots, a_{n, m} \rangle_{\mathcal{B}''}
  \end{aligned}\]

  And define \[f\restriction_{\mathcal{B}, \mathcal{B'}} =  \begin{array}{|ccc|}
    a_{1,1} & \dots & a_{1,m}\\ \vdots & & \vdots\\ a_{n,1} & \dots & a_{n, m}
  \end{array}\] the matricial representation of $f$ w.r.t. $\mathcal{B}, \mathcal{B'}$
  
\end{frame}

\begin{frame}{Compatibility with previous concepts.}
  Already a representation for Linear maps $\mathbb{R}^m \mapsto \mathbb{R}^n$. We actually generalized this:

  Let $\mathcal{E}^{(m)}= \langle e^{(m)}_1, \dots, e^{(m)}_m \rangle$ be the canonical basis of $\mathbb{R}^m$ and $\mathcal{E}^{(n)} = \langle e^{(n)}_1, \dots, e^{(n)}_n \rangle$ be the canonical basis of $\mathbb{R}^n$.
  Let $f: \mathbb{R}^m \mapsto \mathbb{R}^n$ be a linear map and $A$ be a $(n,m)$ matrix. Then \[\widehat{A} = f \textnormal{ iff } A = f\restriction_{\mathcal{E}^{(m)}, \mathcal{E}^{(n)}}\].
\end{frame}



\begin{frame}
  \frametitle{Example}
  \begin{exampleblock}{A linear map}
    \[f: \begin{aligned} \mathbb{R}^2 &\mapsto \mathbb{R}^2\\ (x,y) &\mapsto (x - y, y - x) \end{aligned}\]

    Two basis of $\mathbb{R}^2 :$
    \[\begin{aligned}
    &\mathcal{E} = \langle e_1, e_2 \rangle = \langle (1,0), (0,1) \rangle \\ 
    &\mathcal{B} = \langle \alpha, \beta \rangle = \langle (1,1), (-1, 1) \rangle\\
    \end{aligned}
    \]
    \[
    \begin{aligned}
      &f(e_1) = (1,-1) = {\color{blue} \langle 1, -1 \rangle_{\mathcal{E}}} = {\color{darkgreen} \langle 0, -1\rangle_{\mathcal{B}}}\\
      &f(e_2) = (-1, 1) = {\color{blue} \langle -1, 1 \rangle_{\mathcal{E}}} = {\color{darkgreen} \langle 0, 1 \rangle_{\mathcal{B}}} \\      
    \end{aligned}
    \]


    \[f\restriction_{{\color{red}\mathcal{E}}, {\color{blue}\mathcal{E}}}:
    \begin{array}{|ccc|}
      f({\color{red}e_1}) & f({\color{red}e_2}) & \\
      \color{blue}\phantom{-}1 & \color{blue} -1 & \color{blue} e_1\\
      \color{blue}-1 & \color{blue} \phantom{-}1 & \color{blue} e_2
    \end{array}
    \quad
    f\restriction_{{\color{red}\mathcal{E}}, {\color{darkgreen} \mathcal{B}}}
    \begin{array}{|ccc|}
      f({\color{red}e_1}) & f({\color{red}e_2}) & \\
      \color{darkgreen}\phantom{-}0 & \color{darkgreen} \phantom{-}0 & \color{darkgreen} \alpha\\
      \color{darkgreen} -1 & \color{darkgreen}\phantom{-}1 & \color{darkgreen} \beta
    \end{array}
    \]

    
  \end{exampleblock}
\end{frame}

\begin{frame}{Reinterpreting matrix application}
  \begin{itemize}
  \item Let $V, W$ be two vector spaces of dimension $m$ and $n$ respectively.
  \item Let $f: V \mapsto W$ be linear.
  \item Let $\mathcal{B}, \mathcal{B}'$ basis of $V$ and $W$ respectively.
  \item Let $v \in V$, $v = \langle x_1, \dots, x_m \rangle_{\mathcal{B}}$.
  \item Let $F \in \mathcal{M}(m, n)$ such that $F = f\restriction_{\mathcal{B}, \mathcal{B}'}$.
  \end{itemize}
  Then
  \[F \times \begin{pmatrix} x_1 \\ \vdots \\ x_m \end{pmatrix} = \begin{pmatrix} y_1 \\\vdots\\y_n \end{pmatrix} \textnormal{ iff } f(v) = \langle y_1, \dots, y_n \rangle_{\mathcal{B}'} \]
\end{frame}

\begin{frame}{Example}
  \begin{exampleblock}{A linear map}
    \[f: \begin{aligned} \mathbb{R}^2 &\mapsto \mathbb{R}^2\\ (x,y) &\mapsto (x - y, y - x) \end{aligned}\]

    Two basis of $\mathbb{R}^2 :$
    \[\begin{aligned}
    &\mathcal{E} = \langle e_1, e_2 \rangle = \langle (1,0), (0,1) \rangle \\ 
    &\mathcal{B} = \langle \alpha, \beta \rangle = \langle (1,1), (-1, 1) \rangle\\
    \end{aligned}
    \]
    \[
    \begin{aligned}
      &f(e_1) = (1,-1) = {\color{blue} \langle 1, -1 \rangle_{\mathcal{E}}} = {\color{darkgreen} \langle 0, -1\rangle_{\mathcal{B}}}\\
      &f(e_2) = (-1, 1) = {\color{blue} \langle -1, 1 \rangle_{\mathcal{E}}} = {\color{darkgreen} \langle 0, 1 \rangle_{\mathcal{B}}} \\      
    \end{aligned}
    \]


    \[
    \underbrace{
    \begin{pmatrix}
      0 & 0 \\
      -1 & 1
    \end{pmatrix}
    }_{f\restriction_{\mathcal{E}, \mathcal{B}}}
    \times
    \underbrace{
    \begin{pmatrix}
      2\\
      1
    \end{pmatrix}
    }_{v = (2,1) = \langle 2, 1\rangle_{\mathcal{E}}}
    =
    \underbrace{
      \begin{pmatrix}
        0\\
        -1
      \end{pmatrix}
    }_{f(v) = (1, -1) = \langle 0,-1 \rangle_{\mathcal{B}}}
    \]  
  \end{exampleblock}
\end{frame}


\begin{frame}
  \frametitle{Reinterpreting product of matrices:}
  \begin{block}{Theorem}
    \begin{itemize}
    \item Let $U, V, W$ be three vector spaces of dimension $l, m$ and $n$ respectively.
    \item Let $g: U \mapsto V$ linear, and $f: V \mapsto W$ linear.
    \item Let $\mathcal{B}, \mathcal{B}'$ and $\mathcal{B}''$ be basis of $U, V$ and $W$ respectively.
    \item Let $G \in \mathcal{M}(l, m)$, $F \in \mathcal{M}(m, n)$ such that $G = g\restriction_{\mathcal{B}, \mathcal{B}'}$ and $F = f\restriction_{\mathcal{B'}, \mathcal{B}''}$.
    \end{itemize}

    Then
    \[F \times G =  f \circ g \restriction{\mathcal{B}, \mathcal{B}''}\]
    \end{block}
    
\end{frame}

\begin{frame}{Example}
  \begin{exampleblock}{Change of basis}
    \[f: \begin{aligned} \mathbb{R}^2 &\mapsto \mathbb{R}^2\\ (x,y) &\mapsto (x - y, y - x) \end{aligned} \quad id_{\mathbb{R}^2}: \begin{aligned} \mathbb{R}^2 &\mapsto \mathbb{R}^2\\ (x,y) &\mapsto (x , y) \end{aligned}\]

    Two basis of $\mathbb{R}^2 :$
    \[\begin{aligned}
    &\mathcal{E} = \langle e_1, e_2 \rangle = \langle (1,0), (0,1) \rangle \\ 
    &\mathcal{B} = \langle \alpha, \beta \rangle = \langle (1,1), (-1, 1) \rangle\\
    \end{aligned}\]

    \[\underbrace{\begin{pmatrix} 1/2& 1/2 \\ -1/2 & 1/2  \end{pmatrix}}_{id_{\mathbb{R}^2} \restriction_{\mathcal{E}, \mathcal{B}}} \times \underbrace{\begin{pmatrix} 1 & -1 \\ -1 & 1 \end{pmatrix}}_{f \restriction{\mathcal{E}}, \mathcal{E}} = \underbrace{\begin{pmatrix} 0 & 0\\ -1 & 1 \end{pmatrix}}_{(id_{\mathbb{R}^2} \circ f) = f \restriction_{\mathcal{E}, \mathcal{B}}}  \]
  \end{exampleblock}
\end{frame}


\begin{frame}{Change of basis}
  \begin{block}{Representations of the identity.}
    \begin{itemize}
    \item  Let $V$ be a vector space of dim $n$.
    \item Let $id_{V} : v \mapsto v$.
    \item Let $\mathcal{B}$, $\mathcal{B}'$ be two basis of $V$.
    \item The $(n,n)$ matrix $P_{\mathcal{B}, \mathcal{B}'} = id_{V} \restriction_{\mathcal{B}, \mathcal{B'}}$ is called the \emph{change of basis} from $\mathcal{B}$ to $\mathcal{B}'$.
    \end{itemize}
    If $v = \langle x_1, \dots, x_n \rangle_{\mathcal{B}}$, we have:
    \[ P_{\mathcal{B}, \mathcal{B}'} \times \begin{pmatrix} x_1 \\\vdots\\x_n \end{pmatrix} = \begin{pmatrix} y_1 \\\vdots\\y_n \end{pmatrix} \textnormal{ iff } v = \langle y_1, \dots, y_n \rangle_{\mathcal{B}'}  \]
    
  \end{block}  
\end{frame}

\begin{frame}
  \frametitle{Particular case}
  \[P_{\mathcal{B}, \mathcal{B}} \textnormal{ is the } (n,n) \textnormal{ identity matrix. } \]
\end{frame}

\end{document}
