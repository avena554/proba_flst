
\documentclass{beamer}

%\usepackage{listings}
%\usepackage[francais]{babel}
\usepackage[T1]{fontenc}
\usepackage[utf8]{inputenc}
%\usepackage{MyriadPro}
\usepackage{cabin}
\usepackage{graphicx}
\usepackage{array}
\usepackage{tikz}
\usetikzlibrary{positioning, backgrounds, shapes, chains, decorations.pathmorphing}

\usepackage{amsmath,amsthm,amssymb}  
\usepackage{stmaryrd}
%\usepackage{mdsymbol}
\usepackage{MnSymbol}
\usepackage{xcolor}
\usepackage{verbatim}
\usepackage{array}
%\usepackage{csquotes}



\usepackage[absolute,overlay]{textpos}
%\usepackage[texcoord,
%grid,gridcolor=red!10,subgridcolor=green!10,gridunit=pt]
%{eso-pic}



\useoutertheme{infolines}

\newcommand{\hidden}[1]{}

%colors
\definecolor{darkgreen}{rgb}{0,0.5,0}
\usebeamercolor{block title}
\definecolor{beamerblue}{named}{fg}
\usebeamercolor{alert block title}
\definecolor{beamealert}{named}{fg}

\renewcommand{\colon}{\!:\!}


\newcommand\paraitem{%
 \quad
 \makebox[\labelwidth][r]{%
 \makelabel{%
 \usebeamertemplate{itemize \beameritemnestingprefix item}}}\hskip\labelsep}

\newcommand{\mmid}{\mathbin{{\mid}{\mid}}}

\begin{document}

\title{Linear algebra -- introduction.} 
\author{Antoine Venant}
%\institute{UDS COLI}
\date{\today}
\maketitle

\begin{frame}
  \frametitle{Introduction.}
  \begin{itemize}
  \item \emph{Linear algebra} is a branch of mathematics.
  \item Deals with specific kind of mathematical spaces and transformations thereof -- namely vector spaces and linear maps.
  \item So... linear algebra deals with linear stuff. Hum.
  \item In fact, ubiquitous in mathematics.
  \item Today: informal presentation of the main concepts and intuitions.
  \end{itemize}
\end{frame}

\begin{frame}
  \frametitle{A first vector space.}
  \begin{exampleblock}{The real line.}
    A common way to display the set of real numbers is by mean of a continuous line:
    \begin{center}
      \begin{overprint}
        \onslide<1>\begin{tikzpicture}[]
\draw[very thick, white] (1,0) -- (2,0) node[midway, above]{\phantom{$\bar 1$}};
%\path [draw=black, fill=black] (1,0) circle (2pt);
%\path [draw=black, fill=white, thick] (2,0.0) circle (2pt);
  \draw[latex-latex] (-5,0) -- (5,0) ;
  \foreach \x in  {-3,-2,-1,0,1,2,3}
  \draw[shift={(\x,0)},color=black] (0pt,3pt) -- (0pt,-3pt);
  \foreach \x in {-3,-2,-1,0,1,2,3}
  \draw[shift={(\x,0)},color=black] (0pt,0pt) -- (0pt,-3pt) node[below] 
       {$\x$};
       
\draw[shift={(3.141592,0)},color=black] (0pt,3pt) -- (0pt,-3pt);
\draw[shift={(3.141592,0)},color=black] (0pt,0pt) -- (0pt,-3pt) node[below, xshift=3] 
     {$\pi$};

\draw[shift={(2.71828,0)},color=black] (0pt,3pt) -- (0pt,-3pt);
\draw[shift={(2.71828,0)},color=black] (0pt,0pt) -- (0pt,-3pt) node[below] 
     {$e$};

     %\draw[shift={(1.414,0)},color=black] (0pt,3pt) -- (0pt,-3pt);
     %\draw[shift={(1.414,0)},color=black] (0pt,0pt) -- (0pt,-3pt) node[below] 
     %     {$\sqrt(2)$};
\end{tikzpicture}

        \onslide<2->\begin{tikzpicture}[]
  \draw[very thick, ->, blue] (0,0) -- (1,0) node[midway, above]{\color{blue}$\overline 1$};
  \draw[very thick, ->, red] (1,0) -- (2,0) node[midway, above]{\color{red}$\overline 1$};
  \draw[very thick, ->, darkgreen] (0,0) -- (-1,0) node[midway, above]{\color{darkgreen}$\overline{-1}$};
%\path [draw=black, fill=black] (1,0) circle (2pt);
%\path [draw=black, fill=white, thick] (2,0.0) circle (2pt);
\draw[latex-latex] (-5,0) -- (5,0) ;
\foreach \x in  {-3,-2,-1,0,1,2,3}
\draw[shift={(\x,0)},color=black] (0pt,3pt) -- (0pt,-3pt);
\foreach \x in {-3,-2,0,3}
\draw[shift={(\x,0)},color=black] (0pt,0pt) -- (0pt,-3pt) node[below] 
     {$\x$};


\draw[shift={(-1,0)},color=black] (0pt,3pt) -- (0pt,-3pt);
\draw[shift={(-1,0)},color=black] (0pt,0pt) -- (0pt,-3pt) node[below, xshift=3] 
     {${\color{darkgreen} \overline{-1}(0)}$};
     
\draw[shift={(1,0)},color=black] (0pt,3pt) -- (0pt,-3pt);
\draw[shift={(1,0)},color=black] (0pt,0pt) -- (0pt,-3pt) node[below, xshift=3] 
     {${\color{blue} \overline{1}(0)}$};

    
\draw[shift={(2,0)},color=black] (0pt,3pt) -- (0pt,-3pt);
\draw[shift={(2,0)},color=black] (0pt,0pt) -- (0pt,-3pt) node[below, xshift=3] 
     {${\color{red} \overline{1}(1)}$};
     
\draw[shift={(3.141592,0)},color=black] (0pt,3pt) -- (0pt,-3pt);
\draw[shift={(3.141592,0)},color=black] (0pt,0pt) -- (0pt,-3pt) node[below, xshift=3] 
     {$\pi$};

\draw[shift={(2.71828,0)},color=black] (0pt,3pt) -- (0pt,-3pt);
\draw[shift={(2.71828,0)},color=black] (0pt,0pt) -- (0pt,-3pt) node[below] 
     {$e$};

     %\draw[shift={(1.414,0)},color=black] (0pt,3pt) -- (0pt,-3pt);
     %\draw[shift={(1.414,0)},color=black] (0pt,0pt) -- (0pt,-3pt) node[below, yshift=-1pt] 
     %{$\sqrt(2)$};
\end{tikzpicture}

      \end{overprint}
    \end{center}
    \begin{itemize}
    \item<1-> Each number corresponds to a position on the line.
    \item<2-> Associate each number with a \emph{vector}.
    \item<3-> Vector $\overline x$ represents a translation of a point towards right or left of length $|x|$ along the line.
    \item<4-> Right if $x$ positive, left if $x$ negative 
    \item<5-> $\overline x(z)$: result of translating $z$ with $\overline x$. 
    \end{itemize}
  \end{exampleblock}
\end{frame}

\begin{frame}
  \frametitle{Vector addition.}
  \begin{exampleblock}{$\overline{3} = \overline{1 + 2}$}
    \begin{center}
      \begin{tikzpicture}[]
  \draw[very thick, ->, darkgreen, yshift=20pt] (0,0) -- (3,0) node[midway, above]{\color{darkgreen}$\overline 3$};
  \draw[very thick, ->, red] (0,0) -- (2,0) node[midway, above]{\color{red}$\overline 2$};
  \draw[very thick, ->, blue] (2,0) -- (3,0) node[midway, above]{\color{blue}$\overline 1$};

  
  %\draw[very thick, ->, darkgreen] (0,0) -- (-1,0) node[midway, above]{\color{darkgreen}$\overline{-1}$};
%\path [draw=black, fill=black] (1,0) circle (2pt);
%\path [draw=black, fill=white, thick] (2,0.0) circle (2pt);
\draw[latex-latex] (-5,0) -- (5,0) ;
\foreach \x in  {-3,-2,-1,0,1,2,3}
\draw[shift={(\x,0)},color=black] (0pt,3pt) -- (0pt,-3pt);
\foreach \x in {-3,-2,-1,0,1,2}
\draw[shift={(\x,0)},color=black] (0pt,0pt) -- (0pt,-3pt) node[below] 
     {$\x$};


%\draw[shift={(-1,0)},color=black] (0pt,3pt) -- (0pt,-3pt);
%\draw[shift={(-1,0)},color=black] (0pt,0pt) -- (0pt,-3pt) node[below, xshift=3] 
%     {${\color{darkgreen} \overline{-1}(0)}$};
     
%\draw[shift={(1,0)},color=black] (0pt,3pt) -- (0pt,-3pt);
%\draw[shift={(1,0)},color=black] (0pt,0pt) -- (0pt,-3pt) node[below, xshift=3] 
%     {${\color{blue} \overline{1}(0)}$};

   
\draw[shift={(3,0)},color=black] (0pt,3pt) -- (0pt,-3pt);
\draw[shift={(3,0)},color=black] (0pt,0pt) -- (0pt,-3pt) node[below] 
     {$\overline{1}(\overline{2}(0)$};

\end{tikzpicture}

\begin{tikzpicture}[]
  \draw[very thick, ->, darkgreen, yshift=20pt] (0,0) -- (3,0) node[midway, above]{\color{darkgreen}$\overline 3$};
  \draw[very thick, ->, red] (1,0) -- (3,0) node[midway, above]{\color{red}$\overline 2$};
  \draw[very thick, ->, blue] (0,0) -- (1,0) node[midway, above]{\color{blue}$\overline 1$};

  
  %\draw[very thick, ->, darkgreen] (0,0) -- (-1,0) node[midway, above]{\color{darkgreen}$\overline{-1}$};
%\path [draw=black, fill=black] (1,0) circle (2pt);
%\path [draw=black, fill=white, thick] (2,0.0) circle (2pt);
\draw[latex-latex] (-5,0) -- (5,0) ;
\foreach \x in  {-3,-2,-1,0,1,2,3}
\draw[shift={(\x,0)},color=black] (0pt,3pt) -- (0pt,-3pt);
\foreach \x in {-3,-1,-2,0,1,2}
\draw[shift={(\x,0)},color=black] (0pt,0pt) -- (0pt,-3pt) node[below] 
     {$\x$};


%\draw[shift={(-1,0)},color=black] (0pt,3pt) -- (0pt,-3pt);
%\draw[shift={(-1,0)},color=black] (0pt,0pt) -- (0pt,-3pt) node[below, xshift=3] 
%     {${\color{darkgreen} \overline{-1}(0)}$};
     
%\draw[shift={(1,0)},color=black] (0pt,3pt) -- (0pt,-3pt);
%\draw[shift={(1,0)},color=black] (0pt,0pt) -- (0pt,-3pt) node[below, xshift=3] 
%     {${\color{blue} \overline{1}(0)}$};

    
\draw[shift={(3,0)},color=black] (0pt,3pt) -- (0pt,-3pt);
\draw[shift={(3,0)},color=black] (0pt,0pt) -- (0pt,-3pt) node[below, xshift=3] 
     {$\overline{2}(\overline{1}(0))$};
     
   
\end{tikzpicture}



    \end{center}
    
    \begin{itemize}
    \item Define $\overline x + \overline y = \overline{x + y}$.
    \item \alert{adding vectors corresponds to composing translations!}
    \item $\overline{x + y}(z) = \overline{x}(\overline{y}(z)) = \overline{y}(\overline{x}(z))$.
    %\item \emph{Moving of $x$ steps to the right then $z$ steps to the right} vs. \emph{Moving from $x + z$ steps} to the right. 
    \end{itemize}
  \end{exampleblock}
\end{frame}

\begin{frame}
  \frametitle{outer law.}
    \begin{center}
      \begin{tikzpicture}[]
  \draw[very thick, ->, darkgreen, yshift=20pt] (0,0) -- (2,0) node[midway, above]{\color{darkgreen}$\overline 2 = 2 \overline 1 = \overline 1 + \overline 1$};

  \draw[very thick, ->, blue] (0,0) -- (1,0) node[midway, above]{\color{blue}$\overline 1$};
  \draw[very thick, ->, blue] (1,0) -- (2,0) node[midway, above]{\color{blue}$\overline 1$};
 
  
  %\draw[very thick, ->, darkgreen] (0,0) -- (-1,0) node[midway, above]{\color{darkgreen}$\overline{-1}$};
%\path [draw=black, fill=black] (1,0) circle (2pt);
%\path [draw=black, fill=white, thick] (2,0.0) circle (2pt);
\draw[latex-latex] (-5,0) -- (5,0) ;
\foreach \x in  {-3,-2,-1,0,1,2,3}
\draw[shift={(\x,0)},color=black] (0pt,3pt) -- (0pt,-3pt);
\foreach \x in {-3,-2,-1,0,1,2,3}
\draw[shift={(\x,0)},color=black] (0pt,0pt) -- (0pt,-3pt) node[below] 
     {$\x$};


%\draw[shift={(-1,0)},color=black] (0pt,3pt) -- (0pt,-3pt);
%\draw[shift={(-1,0)},color=black] (0pt,0pt) -- (0pt,-3pt) node[below, xshift=3] 
%     {${\color{darkgreen} \overline{-1}(0)}$};
     
%\draw[shift={(1,0)},color=black] (0pt,3pt) -- (0pt,-3pt);
%\draw[shift={(1,0)},color=black] (0pt,0pt) -- (0pt,-3pt) node[below, xshift=3] 
%     {${\color{blue} \overline{1}(0)}$};

  

\end{tikzpicture}

    \end{center}
    
    \begin{itemize}
    \item Define $\alpha\overline x = \overline{\alpha x }$.
    \item $2\overline{x} = \overline 2x = \overline{x} + \overline{x}$
    \item For any $x$, we have $\bar x = x \bar 1$
    \item \alert{$\langle \overline 1 \rangle$ is a \emph{basis} of the vector space!}
    \end{itemize}
\end{frame}

\begin{frame}{A $2D$ space.}
  \begin{itemize}
  \item Previous space has a one vector basis $\rightarrow$ dimension $1$.
  \item What if instead of a single continuous line, we consider $2$ of them?
  \end{itemize}

  \begin{exampleblock}{The plan $\mathbb{R}^2$.}
    \begin{center}
      \begin{tikzpicture}[scale = 0.6]
\draw[help lines, color=gray!30, dashed] (-3.9,-3.9) grid (3.9,3.9);
\draw[->,ultra thick] (-4,0)--(4,0) node[right]{$x$};
\draw[->,ultra thick] (0,-4)--(0,4) node[above]{$y$};

\foreach \x in  {-3,-2,-1,0,1,2,3}
\draw[shift={(\x,0)},color=black] (0pt,3pt) -- (0pt,-3pt);
\foreach \x in {-3,-2,-1,0,1,2,3}
\draw[shift={(\x,0)},color=black] (0pt,0pt) -- (0pt,-3pt) node[below] 
     {$\x$};


\foreach \y in  {-3,-2,-1,1,2,3}
\draw[shift={(0, \y)},color=black] (3pt,0pt) -- (-3pt,-0pt);
\foreach \y in {-3,-2,-1,1,2,3}
\draw[shift={(0, \y)},color=black] (0pt,0pt) -- (-3pt,0pt) node[right] 
     {$\y$};
     

\draw[red] (3,2) node[ultra thick]{\textbullet} node[below]{$(3,2)$};

\end{tikzpicture}

    \end{center}
  \end{exampleblock}
\end{frame}

\begin{frame}{$2D$ vectors}
  \begin{itemize}
  \item Again, let's associate each point $(x,y)$ with a \emph{vector} $\overline {(x,y)}$.
  \item The points $(0,0)$ and $(x,y)$ define a unique line.
  \item And a unique orientation of that line.
  \end{itemize}

  \begin{center}
    \begin{tikzpicture}[scale = 0.6]
\draw[help lines, color=gray!30, dashed] (-3.9,-3.9) grid (3.9,3.9);
\draw[->,ultra thick] (-4,0)--(4,0) node[right]{$x$};
\draw[->,ultra thick] (0,-4)--(0,4) node[above]{$y$};

\foreach \x in  {-3,-2,-1,0,1,2,3}
\draw[shift={(\x,0)},color=black] (0pt,3pt) -- (0pt,-3pt);
\foreach \x in {-3,-2,-1,0,1,2,3}
\draw[shift={(\x,0)},color=black] (0pt,0pt) -- (0pt,-3pt) node[below] 
     {$\x$};


\foreach \y in  {-3,-2,-1,1,2,3}
\draw[shift={(0, \y)},color=black] (3pt,0pt) -- (-3pt,-0pt);
\foreach \y in {-3,-2,-1,1,2,3}
\draw[shift={(0, \y)},color=black] (0pt,0pt) -- (-3pt,0pt) node[right] 
     {$\y$};
     

     \draw[red] (3,2) node[ultra thick]{\textbullet} node[below]{$(3,2)$};
     \draw[blue, ->, ultra thick] (0,0) -- (3,2) node[midway, below]{$\overline{(3,2)}$};

\end{tikzpicture}

  \end{center}

\end{frame}

\begin{frame}{$2D$ vectors.}
  \begin{itemize}
  \item Again, interpret vectors as \emph{translations}.
  \item \emph{Translate} $(\alpha, \beta)$ along $v = \overline{(x, y)}$: 
    \begin{itemize}
    \item Move $\alpha$ along a line and orientation parallel to $(0,0) -- (x,y)$.
    \item Move on a path of the same length separating $(x, y)$ from $(0,0)$.
    \item equivalently: $\overline{(x,y)}(\alpha, \beta) = (\alpha + x, \beta + y)$
    \end{itemize}
  \end{itemize}
  
  \begin{center}
    \begin{tikzpicture}[scale = 0.6]
\draw[help lines, color=gray!30, dashed] (-3.9,-3.9) grid (3.9,3.9);
\draw[->,ultra thick] (-4,0)--(4,0) node[right]{$x$};
\draw[->,ultra thick] (0,-4)--(0,4) node[above]{$y$};

\foreach \x in  {-3,-2,-1,0,1,2,3}
\draw[shift={(\x,0)},color=black] (0pt,3pt) -- (0pt,-3pt);
\foreach \x in {-3,-2,-1,0,1,2,3}
\draw[shift={(\x,0)},color=black] (0pt,0pt) -- (0pt,-3pt) node[below] 
     {$\x$};


\foreach \y in  {-3,-2,-1,1,2,3}
\draw[shift={(0, \y)},color=black] (3pt,0pt) -- (-3pt,-0pt);
\foreach \y in {-3,-2,-1,1,2,3}
\draw[shift={(0, \y)},color=black] (0pt,0pt) -- (-3pt,0pt) node[right] 
     {$\y$};
     

     \draw[red] (3,2) node[ultra thick]{\textbullet} node[below]{$(3,2)$};
     \draw[blue, ->, ultra thick] (0,0) -- (3,2) node[midway, below]{$\overline{(3,2)}$};
     \draw[blue, ->, ultra thick] (1,2) -- (4,4) node[right]{$\overline{(3,2)}(1,2)$};

\end{tikzpicture}

  \end{center}
  
\end{frame}

\begin{frame}{Adding $2D$ vectors.}
  \begin{itemize}
  \item As in $1D$: $\overline{(x, y)} + \overline{(z, t)} = \overline{(x + y, z + t)}.$
  \item As in $1D$ vector addition corresponds to composing translations.
  \item $\overline{(x, y)} + \overline{(z, t)}(\alpha, \beta) = \overline{(x, y)}(\overline{(z, t)}(\alpha, \beta))$
  \end{itemize}

  \begin{center}
    \begin{tikzpicture}[scale = 0.6]
\draw[help lines, color=gray!30, dashed] (-3.9,-3.9) grid (3.9,3.9);
\draw[->,ultra thick] (-4,0)--(4,0) node[right]{$x$};
\draw[->,ultra thick] (0,-4)--(0,4) node[above]{$y$};

\foreach \x in  {-3,-2,-1,0,1,2,3}
\draw[shift={(\x,0)},color=black] (0pt,3pt) -- (0pt,-3pt);
\foreach \x in {-3,-2,-1,0,1,2,3}
\draw[shift={(\x,0)},color=black] (0pt,0pt) -- (0pt,-3pt) node[below] 
     {$\x$};


\foreach \y in  {-3,-2,-1,1,2,3}
\draw[shift={(0, \y)},color=black] (3pt,0pt) -- (-3pt,-0pt);
\foreach \y in {-3,-2,-1,1,2,3}
\draw[shift={(0, \y)},color=black] (0pt,0pt) -- (-3pt,0pt) node[right] 
     {$\y$};
     

     %\draw[red] (3,2) node[ultra thick]{\textbullet} node[below]{$(3,2)$};
     \draw[blue, ->, ultra thick] (0,0) -- (3,2);
     \draw[blue, ->, ultra thick] (1,2) -- (4,4);
     \draw[red, ->, ultra thick] (0,0) -- (1,2);
     \draw[red,->, ultra thick] (3,2) -- (4,4);
     \draw[darkgreen, ->, ultra thick] (0,0) -- (4,4) node[right]{$\overline{(3,2)}+\overline{(1,2)}$};

\end{tikzpicture}

  \end{center}
 
\end{frame}


\begin{frame}
  \frametitle{Outer law with $2D$ vectors.}
  \begin{itemize}
  \item definition for $\alpha \overline{(x,y)}$?
  \item As in $1D$: `stretch' the translation to a length of $\alpha$ times its previous length.
  \item $\alpha \overline{(x,y)} = \overline{(\alpha x, \alpha y)}$.
  \end{itemize}


\begin{center}
    \begin{tikzpicture}[scale = 0.6]
\draw[help lines, color=gray!30, dashed] (-3.9,-3.9) grid (3.9,3.9);
\draw[->,ultra thick] (-4,0)--(4,0) node[right]{$x$};
\draw[->,ultra thick] (0,-4)--(0,4) node[above]{$y$};

\foreach \x in  {-3,-2,-1,0,1,2,3}
\draw[shift={(\x,0)},color=black] (0pt,3pt) -- (0pt,-3pt);
\foreach \x in {-3,-2,-1,0,1,2,3}
\draw[shift={(\x,0)},color=black] (0pt,0pt) -- (0pt,-3pt) node[below] 
     {$\x$};


\foreach \y in  {-3,-2,-1,1,2,3}
\draw[shift={(0, \y)},color=black] (3pt,0pt) -- (-3pt,-0pt);
\foreach \y in {-3,-2,-1,1,2,3}
\draw[shift={(0, \y)},color=black] (0pt,0pt) -- (-3pt,0pt) node[right] 
     {$\y$};
     
     
     \draw[darkgreen, ->, ultra thick] (0,0) -- (3, 3) node[right, near end]{$3\overline{(1,1)}$};
     \draw[blue, ->, dashed, ultra thick] (0,0) -- (1, 1) node[right, near end]{$\overline{(1,1)}$};
     
\end{tikzpicture}

  \end{center}
\end{frame}

\begin{frame}
  \frametitle{Basis.}
  \begin{itemize}
  \item Any vector $\overline{(x,y)}$ can be uniquely written as:
  \item $\overline{(x,y)} = x\overline{(1,0)} + y \overline{(0,1)}$.
  \item \alert{$\langle (0,1), (1,0) \rangle$ is a basis\footnote{called the \emph{canonical} basis} of $\mathbb{R}^2$.}
  \end{itemize}

  \begin{center}
    \begin{tikzpicture}[scale = 0.7]
\draw[help lines, color=gray!30, dashed] (-2.9,-2.9) grid (2.9,2.9);
\draw[->,ultra thick] (-3,0)--(3,0) node[right]{$x$};
\draw[->,ultra thick] (0,-3)--(0,3) node[above]{$y$};

\foreach \x in  {-2,-1,0,1,2}
\draw[shift={(\x,0)},color=black] (0pt,3pt) -- (0pt,-3pt);
\foreach \x in {-2,-1,0,1,2}
\draw[shift={(\x,0)},color=black] (0pt,0pt) -- (0pt,-3pt) node[below] 
     {$\x$};


\foreach \y in  {-2,-1,1,2}
\draw[shift={(0, \y)},color=black] (3pt,0pt) -- (-3pt,-0pt);
\foreach \y in {-2,-1,1,2}
\draw[shift={(0, \y)},color=black] (0pt,0pt) -- (-3pt,0pt) node[right] 
     {$\y$};
     
     
     \draw[blue, ->, ultra thick, dashed] (0,0) -- (0, 1);
     \draw[blue, ->, ultra thick, dashed] (0,0) -- (1, 0);
     \draw[red, ->, ultra thick] (0,0) -- (0.666666,0);
     \draw[red, ->, ultra thick] (0,0) -- (0,2);
     \draw[red, ->, thick, dashed] (0.66666666,0) -- (0.66666666,2);
     \draw[red, ->, thick, dashed] (0,2) -- (0.666666666666,2);
     \draw[darkgreen, ->, ultra thick] (0,0) -- (0.6666666666,2) node[above right]{$\frac{2}{3}\overline{(1,0)} + 2\overline{(0,1)}$};     
     
\end{tikzpicture}

  \end{center}
  
\end{frame}

\begin{frame}
  \frametitle{Basis.}
  \begin{itemize}
  \item (Informally) basis: sequence of vectors $B = \langle e_1, e_2,\dots, e_n \rangle$ s.t. any vector $v$ has a unique decomposition $v = \alpha_1 e_1 + \dots +\alpha_n e_n$, with $\alpha_1, \dots, \alpha_n \in \mathbb{R}$.
  \item A space with a basis of length $n$ has dimension $n$.
  \item $\langle \alpha_1, \dots, \alpha_n \rangle$ is the representation of $v$ in $B$.
  \item In general, there exists more than one basis.
  \end{itemize}

  \begin{exampleblock}{Example}
    \begin{itemize}
    \item $B^{(1)} = \langle\overline 1\rangle$ is a basis of $\mathbb{R}$. $\overline \pi =\pi \overline 1 $ has representation $\langle \pi \rangle$ in $B^{(1)}$.
    \item $B^{(2)} = \langle \overline {(1,0)}, \overline{(0,1)} \rangle$ is a basis of $\mathbb{R}^2$. $(\frac{2}{3},2)$ has representation $\langle \frac{2}{3}, 2 \rangle$ in $B^{(2)}$.
    \end{itemize}
  \end{exampleblock}
\end{frame}

\begin{frame}{Basis.}
  \begin{itemize}
  \item \alert{In general, there exists more than one basis!}
  \end{itemize}

  \begin{exampleblock}{Example}
    \begin{itemize}
    \item $\langle \overline 3 \rangle$ is a basis of $\mathcal{R}$. $\overline{\pi}$ has representation $\langle \pi/3 \rangle$ in this basis.
    \item $B' = \langle \overline {(2,2)}, \overline{(-1,3)}  \rangle$ is a basis of $\mathbb{R}^2$. $\overline{(\frac{2}{3}, 2)}$ has representation $\langle \frac{1}{2}, \frac{1}{3} \rangle$ in $B'$.
    \end{itemize}
    \end{exampleblock}

    \begin{center}
      \begin{tikzpicture}[scale = 0.7]
\draw[help lines, color=gray!30, dashed] (-2.9,-2.9) grid (2.9,2.9);
\draw[->,ultra thick] (-3,0)--(3,0) node[right]{$x$};
\draw[->,ultra thick] (0,-3)--(0,3) node[above]{$y$};

\foreach \x in  {-2,-1,0,1,2}
\draw[shift={(\x,0)},color=black] (0pt,3pt) -- (0pt,-3pt);
\foreach \x in {-2,-1,0,1,2}
\draw[shift={(\x,0)},color=black] (0pt,0pt) -- (0pt,-3pt) node[below] 
     {$\x$};


\foreach \y in  {-2,-1,1,2}
\draw[shift={(0, \y)},color=black] (3pt,0pt) -- (-3pt,-0pt);
\foreach \y in {-2,-1,1,2}
\draw[shift={(0, \y)},color=black] (0pt,0pt) -- (-3pt,0pt) node[right] 
     {$\y$};
     
     
     \draw[blue, ->, ultra thick, dashed] (0,0) -- (2, 2);
     \draw[blue, ->, ultra thick, dashed] (0,0) -- (-1, 3);
     \draw[red, ->, ultra thick] (0,0) -- (1,1);
     \draw[red, ->, ultra thick] (0,0) -- (-0.3333,1);
     \draw[red, ->, thick, dashed] (1,1) -- (0.66666,2);
     \draw[red, ->, thick, dashed] (-0.33333,1) -- (0.66666,2);
     \draw[darkgreen, ->, ultra thick] (0,0) -- (0.66666,2) node[above right]{$\frac{1}{2}\overline{(2,2)} + \frac{1}{3}\overline{(-1,3)}$};     
     
\end{tikzpicture}
      
    \end{center}
  
  \end{frame}

\begin{frame}{Exercise}
  \begin{exampleblock}{$B' \langle \overline{(2,2)}, \overline{(-1, 3)} \rangle$}
    \begin{itemize}
    \item Exercise: which vector has representation $\langle u, v \rangle$ in $B'$?
    %\item Exercise: what is the representation of the vector $\overline {(x,y)}$ in $B'$?
    \end{itemize}
  \end{exampleblock}
\end{frame}

\begin{frame}{Linear equation system.}
  There is a link between changing basis and systems of linear equations:
  A vector $v = \overline{(x, y)}$ is such that:
  \[
  \left \{
  \begin{array}{l}
    \frac{3x + y}{8} = \frac{1}{2}\\
    \frac{y - x}{4} = \frac{1}{3}\\
  \end{array}
  \right .
  \]
  iff $v = \langle \frac{1}{2}, \frac{1}{3} \rangle_{B'}$. Proof?
\end{frame}

\begin{frame}{Linear maps}
 
  (Informally) a linear map is a map between vector spaces such that $f(\alpha u + \beta v) = \alpha f(u) + \beta f(v)$.
  

  \begin{block}{Linear equations and linear map.}
    \begin{itemize}
    \item Exercise: $f( \overline{(x, y)} ) = \overline{(\frac{3x + y}{8}, \frac{y - x}{4} )}$ is a linear map.
    \item {\bf{$f$ is such that $f(\langle u, v \rangle_{B'}) = \langle u, v \rangle_{B^{(2)}} = \overline{(u, v)}$}}
    \item $f$ 'reinterprets' representations from $B'$ in $B^{(2)}$.
    \item So $f(u) = \overline{(a, b)}$ iff $u = \langle a, b \rangle_{B'}$ iff $u = \overline{(2a - b, 2a + 3b)}$.
    \item Considering the `right' basis $B'$ makes computation very easy! We will see how and when we can find it.
    \end{itemize} 
  \end{block}
  
\end{frame}

\begin{frame}{Geometric interpretation.}
  Whiteboard.
\end{frame}

\begin{frame}{Vector space (over $\mathbb{R}$).}
  As necessary ingredients we have:
  \begin{itemize}
  \item Vectors.
  \item Addition of vectors (\emph{inner law}).
  \item Multiplication of vectors by a scalar (\emph{outer law}).
  \item And some specific structure.
  \end{itemize}
\end{frame}

\begin{frame}{Vector space (over $\mathbb{R}$)}
  Formally:
  
  \begin{block}{Definition}
    $\langle V, {+}, \circ \rangle$ is a \emph{vector space} (over $\mathbb{R}$) iff all of the following holds:
    \begin{itemize}
    \item ${+}: V \times V \mapsto V$ and $\langle V, {+} \rangle$ is a commutative group (we let $0$ denote its neutral element).
    \item ${\circ}: \mathbb{R} \times V \mapsto V$.
    \item $\circ(\alpha \beta, u) = \circ(\alpha, \circ(\beta, u))$. (Mixed associativity)
    \item $\circ(1, u) = u$. (Left neutrality)
    \item $\circ(\alpha + \beta, u) = \circ(\alpha, u) + \circ(\beta, u)$. (Right distributivity)
    \item $\circ(\alpha, u+v) = \circ(\alpha, u) + \circ(\beta, v)$. ([Left distributivity)
    \end{itemize}
  \end{block}
\end{frame}



\end{document}
